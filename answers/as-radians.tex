% Options for packages loaded elsewhere
\PassOptionsToPackage{unicode}{hyperref}
\PassOptionsToPackage{hyphens}{url}
\PassOptionsToPackage{dvipsnames,svgnames,x11names}{xcolor}
%
\documentclass[
  12pt,
  a4paper, oneside]{starmastarticle}

\usepackage{amsmath,amssymb}
\usepackage{iftex}
\ifPDFTeX
  \usepackage[T1]{fontenc}
  \usepackage[utf8]{inputenc}
  \usepackage{textcomp} % provide euro and other symbols
\else % if luatex or xetex
  \usepackage{unicode-math}
  \defaultfontfeatures{Scale=MatchLowercase}
  \defaultfontfeatures[\rmfamily]{Ligatures=TeX,Scale=1}
\fi
\usepackage{lmodern}
\ifPDFTeX\else  
    % xetex/luatex font selection
\fi
% Use upquote if available, for straight quotes in verbatim environments
\IfFileExists{upquote.sty}{\usepackage{upquote}}{}
\IfFileExists{microtype.sty}{% use microtype if available
  \usepackage[]{microtype}
  \UseMicrotypeSet[protrusion]{basicmath} % disable protrusion for tt fonts
}{}
\makeatletter
\@ifundefined{KOMAClassName}{% if non-KOMA class
  \IfFileExists{parskip.sty}{%
    \usepackage{parskip}
  }{% else
    \setlength{\parindent}{0pt}
    \setlength{\parskip}{6pt plus 2pt minus 1pt}}
}{% if KOMA class
  \KOMAoptions{parskip=half}}
\makeatother
\usepackage{xcolor}
\usepackage[top=25mm,left=25mm,right=25mm,bottom=25mm]{geometry}
\setlength{\emergencystretch}{3em} % prevent overfull lines
\setcounter{secnumdepth}{-\maxdimen} % remove section numbering
% Make \paragraph and \subparagraph free-standing
\makeatletter
\ifx\paragraph\undefined\else
  \let\oldparagraph\paragraph
  \renewcommand{\paragraph}{
    \@ifstar
      \xxxParagraphStar
      \xxxParagraphNoStar
  }
  \newcommand{\xxxParagraphStar}[1]{\oldparagraph*{#1}\mbox{}}
  \newcommand{\xxxParagraphNoStar}[1]{\oldparagraph{#1}\mbox{}}
\fi
\ifx\subparagraph\undefined\else
  \let\oldsubparagraph\subparagraph
  \renewcommand{\subparagraph}{
    \@ifstar
      \xxxSubParagraphStar
      \xxxSubParagraphNoStar
  }
  \newcommand{\xxxSubParagraphStar}[1]{\oldsubparagraph*{#1}\mbox{}}
  \newcommand{\xxxSubParagraphNoStar}[1]{\oldsubparagraph{#1}\mbox{}}
\fi
\makeatother


\providecommand{\tightlist}{%
  \setlength{\itemsep}{0pt}\setlength{\parskip}{0pt}}\usepackage{longtable,booktabs,array}
\usepackage{calc} % for calculating minipage widths
% Correct order of tables after \paragraph or \subparagraph
\usepackage{etoolbox}
\makeatletter
\patchcmd\longtable{\par}{\if@noskipsec\mbox{}\fi\par}{}{}
\makeatother
% Allow footnotes in longtable head/foot
\IfFileExists{footnotehyper.sty}{\usepackage{footnotehyper}}{\usepackage{footnote}}
\makesavenoteenv{longtable}
\usepackage{graphicx}
\makeatletter
\def\maxwidth{\ifdim\Gin@nat@width>\linewidth\linewidth\else\Gin@nat@width\fi}
\def\maxheight{\ifdim\Gin@nat@height>\textheight\textheight\else\Gin@nat@height\fi}
\makeatother
% Scale images if necessary, so that they will not overflow the page
% margins by default, and it is still possible to overwrite the defaults
% using explicit options in \includegraphics[width, height, ...]{}
\setkeys{Gin}{width=\maxwidth,height=\maxheight,keepaspectratio}
% Set default figure placement to htbp
\makeatletter
\def\fps@figure{htbp}
\makeatother

\usepackage{setspace}
\renewcommand{\familydefault}{cmss}
\renewcommand{\familydefault}{\sfdefault}
\usepackage{multirow}
\usepackage{colortbl}
\usepackage{fancyhdr}
\onehalfspacing
\renewcommand{\arraystretch}{1.2}
\setlength{\parskip}{0.5em}
\setlength{\parindent}{0em}
\newcommand{\mb}[1]{\mathbb{#1}} % blackboard bold
\newcommand{\mc}[1]{\mathcal{#1}} % calligraphic
\newcommand{\mf}[1]{\mathfrak{#1}} % fraktur
\newcommand{\ms}[1]{\mathscr{#1}} % script
\newcommand{\vb}[1]{\mathbf{#1}} % vector bold
\newcommand{\from}{\leftarrow}
\newcommand{\dne}{\hfill \qed \vspace{0.3cm}} % end of proof symbol
\newcommand{\abs}[1]{\left|#1\right|} % modulus signs
\newcommand{\norm}[1]{\left|\left|#1\right|\right|} % norm signs
\renewcommand{\Re}{\mathrm{Re}}
\renewcommand{\Im}{\mathrm{Im}}
\newcommand{\im}{\mathrm{im}}
\newcommand{\ds}{\displaystyle}
\renewcommand{\d}{\mathrm{d}}
\makeatletter
\@ifpackageloaded{caption}{}{\usepackage{caption}}
\AtBeginDocument{%
\ifdefined\contentsname
  \renewcommand*\contentsname{Table of contents}
\else
  \newcommand\contentsname{Table of contents}
\fi
\ifdefined\listfigurename
  \renewcommand*\listfigurename{List of Figures}
\else
  \newcommand\listfigurename{List of Figures}
\fi
\ifdefined\listtablename
  \renewcommand*\listtablename{List of Tables}
\else
  \newcommand\listtablename{List of Tables}
\fi
\ifdefined\figurename
  \renewcommand*\figurename{Figure}
\else
  \newcommand\figurename{Figure}
\fi
\ifdefined\tablename
  \renewcommand*\tablename{Table}
\else
  \newcommand\tablename{Table}
\fi
}
\@ifpackageloaded{float}{}{\usepackage{float}}
\floatstyle{ruled}
\@ifundefined{c@chapter}{\newfloat{codelisting}{h}{lop}}{\newfloat{codelisting}{h}{lop}[chapter]}
\floatname{codelisting}{Listing}
\newcommand*\listoflistings{\listof{codelisting}{List of Listings}}
\makeatother
\makeatletter
\makeatother
\makeatletter
\@ifpackageloaded{caption}{}{\usepackage{caption}}
\@ifpackageloaded{subcaption}{}{\usepackage{subcaption}}
\makeatother

\ifLuaTeX
  \usepackage{selnolig}  % disable illegal ligatures
\fi
\usepackage{bookmark}

\IfFileExists{xurl.sty}{\usepackage{xurl}}{} % add URL line breaks if available
\urlstyle{same} % disable monospaced font for URLs
\hypersetup{
  pdftitle={Answers: Introduction to radians},
  pdfauthor={Ifan Howells-Baines, Mark Toner},
  colorlinks=true,
  linkcolor={blue},
  filecolor={Maroon},
  citecolor={Blue},
  urlcolor={Blue},
  pdfcreator={LaTeX via pandoc}}


\title{Answers: Introduction to radians}
\author{Ifan Howells-Baines, Mark Toner}
\date{}

\begin{document}
\maketitle
\begin{abstract}
Answers to the questions relating to the guide on radians.
\end{abstract}


\emph{These are the answers to
\href{../questions/qs-radians.qmd}{Questions: Introduction to radians}.}

\textbf{Please attempt the questions before reading these answers!}

\subsection*{Q1}\label{q1}
\addcontentsline{toc}{subsection}{Q1}

1.1. Multiplying \(30^\circ\) by \(\pi\) and dividing by \(180\) gives
\(\dfrac{30\pi}{180} \textsf{ rad} = \dfrac{\pi}{6} \textsf{ rad} = 0.524 \textsf{ rad}\)
to three decimal places.

1.2. Multiplying \(105^\circ\) by \(\pi\) and dividing by \(180\) gives
\(\dfrac{105\pi}{180} \textsf{ rad} = \dfrac{7\pi}{12} \textsf{ rad} = 1.833 \textsf{ rad}\)
to three decimal places.

1.3. Multiplying \(298^\circ\) by \(\pi\) and dividing by \(180\) gives
\(\dfrac{298\pi}{180} \textsf{ rad} = \dfrac{149\pi}{90} \textsf{ rad} = 5.201 \textsf{ rad}\)
to three decimal places.

1.4. Multiplying \(61^\circ\) by \(\pi\) and dividing by \(180\) gives
\(\dfrac{61\pi}{180} \textsf{ rad} = 1.064 \textsf{ rad}\) to three
decimal places.

1.5. Multiplying \(353^\circ\) by \(\pi\) and dividing by \(180\) gives
\(\dfrac{353\pi}{180} \textsf{ rad} = 6.161 \textsf{ rad}\) to three
decimal places.

1.6. Multiplying \(197^\circ\) by \(\pi\) and dividing by \(180\) gives
\(\dfrac{197\pi}{180} \textsf{ rad} = 3.438 \textsf{ rad}\) to three
decimal places.

\subsection*{Q2}\label{q2}
\addcontentsline{toc}{subsection}{Q2}

2.1. Multiplying \(\dfrac{\pi}{3} \textsf{ rad}\) by \(180\) and
dividing by \(\pi\) gives \(\dfrac{180\pi}{3\pi} ^\circ = 60 ^\circ\).

2.2. Multiplying \(\dfrac{2\pi}{3}\textsf{ rad}\) by \(180\) and
dividing by \(\pi\) gives \(\dfrac{360\pi}{3\pi} ^\circ = 120 ^\circ\).

2.3. Multiplying \(\dfrac{\pi}{7} \textsf{ rad}\) by \(180\) and
dividing by \(\pi\) gives
\(\dfrac{180\pi}{7\pi} ^\circ = 25.714 ^\circ\) to three decimal places.

2.4. Multiplying \(\dfrac{5\pi}{7}\textsf{ rad}\) by \(180\) and
dividing by \(\pi\) gives
\(\dfrac{900\pi}{7\pi} ^\circ = 128.571 ^\circ\) to three decimal
places.

2.5. Multiplying \(5\textsf{ rad}\) by \(180\) and dividing by \(\pi\)
gives \(\dfrac{900}{\pi} ^\circ = 286.479 ^\circ\) to three decimal
places.

2.6. Multiplying \(\dfrac{3}{4} \textsf{ rad}\) by \(180\) and dividing
by \(\pi\) gives
\(\dfrac{540}{4\pi} ^\circ = \dfrac{135}{\pi} ^\circ = 42.972 ^\circ\)
to three decimal places.

\subsection*{Q3}\label{q3}
\addcontentsline{toc}{subsection}{Q3}

3.1. In this case, the length of the arc is \(\dfrac{7\pi}{8} = 2.749\)
(to 3dp) and the area of the sector is \(\dfrac{49\pi}{16} = 9.621\) (to
3dp).

3.2. In this case, the length of the arc is \(\dfrac{\pi}{2} = 1.571\)
(to 3dp) and the area of the sector is \(\dfrac{\pi}{12} = 0.262\) (to
3dp).

3.3. In this case, the length of the arc is \(14\pi = 43.982\) (to 3dp)
and the area of the sector is \(\dfrac{525\pi}{2} = 824.668\) (to 3dp).

\begin{center}\rule{0.5\linewidth}{0.5pt}\end{center}

\begin{center}\rule{0.5\linewidth}{0.5pt}\end{center}

\subsection*{Version history and
licensing}\label{version-history-and-licensing}
\addcontentsline{toc}{subsection}{Version history and licensing}

v1.0: initial version created 08/23 by Ifan Howells-Baines, Mark Toner
as part of a University of St Andrews STEP project.

\begin{itemize}
\tightlist
\item
  v1.1: edited 05/24 by tdhc.
\end{itemize}

\href{https://creativecommons.org/licenses/by-nc-sa/4.0/?ref=chooser-v1}{This
work is licensed under CC BY-NC-SA 4.0.}




\end{document}
