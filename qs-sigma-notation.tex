% Options for packages loaded elsewhere
\PassOptionsToPackage{unicode}{hyperref}
\PassOptionsToPackage{hyphens}{url}
\PassOptionsToPackage{dvipsnames,svgnames,x11names}{xcolor}
%
\documentclass[
  12pt,
  a4paper, oneside]{starmastarticle}

\usepackage{amsmath,amssymb}
\usepackage{lmodern}
\usepackage{iftex}
\ifPDFTeX
  \usepackage[T1]{fontenc}
  \usepackage[utf8]{inputenc}
  \usepackage{textcomp} % provide euro and other symbols
\else % if luatex or xetex
  \usepackage{unicode-math}
  \defaultfontfeatures{Scale=MatchLowercase}
  \defaultfontfeatures[\rmfamily]{Ligatures=TeX,Scale=1}
\fi
% Use upquote if available, for straight quotes in verbatim environments
\IfFileExists{upquote.sty}{\usepackage{upquote}}{}
\IfFileExists{microtype.sty}{% use microtype if available
  \usepackage[]{microtype}
  \UseMicrotypeSet[protrusion]{basicmath} % disable protrusion for tt fonts
}{}
\makeatletter
\@ifundefined{KOMAClassName}{% if non-KOMA class
  \IfFileExists{parskip.sty}{%
    \usepackage{parskip}
  }{% else
    \setlength{\parindent}{0pt}
    \setlength{\parskip}{6pt plus 2pt minus 1pt}}
}{% if KOMA class
  \KOMAoptions{parskip=half}}
\makeatother
\usepackage{xcolor}
\usepackage[top=25mm,left=25mm,right=25mm,bottom=25mm]{geometry}
\setlength{\emergencystretch}{3em} % prevent overfull lines
\setcounter{secnumdepth}{-\maxdimen} % remove section numbering
% Make \paragraph and \subparagraph free-standing
\ifx\paragraph\undefined\else
  \let\oldparagraph\paragraph
  \renewcommand{\paragraph}[1]{\oldparagraph{#1}\mbox{}}
\fi
\ifx\subparagraph\undefined\else
  \let\oldsubparagraph\subparagraph
  \renewcommand{\subparagraph}[1]{\oldsubparagraph{#1}\mbox{}}
\fi


\providecommand{\tightlist}{%
  \setlength{\itemsep}{0pt}\setlength{\parskip}{0pt}}\usepackage{longtable,booktabs,array}
\usepackage{calc} % for calculating minipage widths
% Correct order of tables after \paragraph or \subparagraph
\usepackage{etoolbox}
\makeatletter
\patchcmd\longtable{\par}{\if@noskipsec\mbox{}\fi\par}{}{}
\makeatother
% Allow footnotes in longtable head/foot
\IfFileExists{footnotehyper.sty}{\usepackage{footnotehyper}}{\usepackage{footnote}}
\makesavenoteenv{longtable}
\usepackage{graphicx}
\makeatletter
\def\maxwidth{\ifdim\Gin@nat@width>\linewidth\linewidth\else\Gin@nat@width\fi}
\def\maxheight{\ifdim\Gin@nat@height>\textheight\textheight\else\Gin@nat@height\fi}
\makeatother
% Scale images if necessary, so that they will not overflow the page
% margins by default, and it is still possible to overwrite the defaults
% using explicit options in \includegraphics[width, height, ...]{}
\setkeys{Gin}{width=\maxwidth,height=\maxheight,keepaspectratio}
% Set default figure placement to htbp
\makeatletter
\def\fps@figure{htbp}
\makeatother

\usepackage{setspace}
\renewcommand{\familydefault}{cmss}
\renewcommand{\familydefault}{\sfdefault}
\usepackage{multirow}
\usepackage{colortbl}
\usepackage{fancyhdr}
\onehalfspacing
\renewcommand{\arraystretch}{1.2}
\setlength{\parskip}{0.5em}
\setlength{\parindent}{0em}
\newcommand{\mb}[1]{\mathbb{#1}} % blackboard bold
\newcommand{\mc}[1]{\mathcal{#1}} % calligraphic
\newcommand{\mf}[1]{\mathfrak{#1}} % fraktur
\newcommand{\ms}[1]{\mathscr{#1}} % script
\newcommand{\vb}[1]{\mathbf{#1}} % vector bold
\newcommand{\from}{\leftarrow}
\newcommand{\dne}{\hfill \qed \vspace{0.3cm}} % end of proof symbol
\newcommand{\abs}[1]{\left|#1\right|} % modulus signs
\newcommand{\norm}[1]{\left|\left|#1\right|\right|} % norm signs
\renewcommand{\Re}{\mathrm{Re}}
\renewcommand{\Im}{\mathrm{Im}}
\newcommand{\im}{\mathrm{im}}
\newcommand{\ds}{\displaystyle}
\makeatletter
\makeatother
\makeatletter
\makeatother
\makeatletter
\@ifpackageloaded{caption}{}{\usepackage{caption}}
\AtBeginDocument{%
\ifdefined\contentsname
  \renewcommand*\contentsname{Table of contents}
\else
  \newcommand\contentsname{Table of contents}
\fi
\ifdefined\listfigurename
  \renewcommand*\listfigurename{List of Figures}
\else
  \newcommand\listfigurename{List of Figures}
\fi
\ifdefined\listtablename
  \renewcommand*\listtablename{List of Tables}
\else
  \newcommand\listtablename{List of Tables}
\fi
\ifdefined\figurename
  \renewcommand*\figurename{Figure}
\else
  \newcommand\figurename{Figure}
\fi
\ifdefined\tablename
  \renewcommand*\tablename{Table}
\else
  \newcommand\tablename{Table}
\fi
}
\@ifpackageloaded{float}{}{\usepackage{float}}
\floatstyle{ruled}
\@ifundefined{c@chapter}{\newfloat{codelisting}{h}{lop}}{\newfloat{codelisting}{h}{lop}[chapter]}
\floatname{codelisting}{Listing}
\newcommand*\listoflistings{\listof{codelisting}{List of Listings}}
\makeatother
\makeatletter
\@ifpackageloaded{caption}{}{\usepackage{caption}}
\@ifpackageloaded{subcaption}{}{\usepackage{subcaption}}
\makeatother
\makeatletter
\@ifpackageloaded{tcolorbox}{}{\usepackage[many]{tcolorbox}}
\makeatother
\makeatletter
\@ifundefined{shadecolor}{\definecolor{shadecolor}{rgb}{.97, .97, .97}}
\makeatother
\makeatletter
\makeatother
\ifLuaTeX
  \usepackage{selnolig}  % disable illegal ligatures
\fi
\IfFileExists{bookmark.sty}{\usepackage{bookmark}}{\usepackage{hyperref}}
\IfFileExists{xurl.sty}{\usepackage{xurl}}{} % add URL line breaks if available
\urlstyle{same} % disable monospaced font for URLs
\hypersetup{
  pdftitle={Sigma Notation: Questions},
  pdfauthor={Ifan Howells-Baines, Mark Toner},
  colorlinks=true,
  linkcolor={blue},
  filecolor={Maroon},
  citecolor={Blue},
  urlcolor={Blue},
  pdfcreator={LaTeX via pandoc}}

\title{Sigma Notation: Questions}
\author{Ifan Howells-Baines, Mark Toner}
\date{}

\begin{document}
\maketitle
\begin{abstract}
Questions relating to the guide on sigma notation
\end{abstract}
\ifdefined\Shaded\renewenvironment{Shaded}{\begin{tcolorbox}[interior hidden, sharp corners, boxrule=0pt, borderline west={3pt}{0pt}{shadecolor}, enhanced, breakable, frame hidden]}{\end{tcolorbox}}\fi

\emph{Before attempting these questions, it is highly recommended that
you read \href{sigmanotation.qmd}{Guide: Sigma Notation}.}

\hypertarget{q1}{%
\subsection*{Q1}\label{q1}}
\addcontentsline{toc}{subsection}{Q1}

Calculate the value of the following. You may use the properties of sums
but they should not be necessary.

1.1 \[\sum_{i = 1}^{10} 2i\]

1.2 \[\sum_{i = 2}^{11} i\]

1.3 \[\sum_{i = 3}^{6} 3i\]

1.4 \[\sum_{i = 1}^{5} i^3\]

1.5 \[\sum_{i = 2}^{6} 5i^2\]

1.6 \[\sum_{i = 3}^{6} 2\]

1.7 \[\sum_{i = 1}^{6} j\]

\hypertarget{q2}{%
\subsection*{Q2}\label{q2}}
\addcontentsline{toc}{subsection}{Q2}

Express the following using sigma notation. Note that there are multiple
correct answers for some of the questions. It is recommended to use i as
your variable so that your answers will align with those provided.

2.1 \(3 + 6 + 9 + 12\)

2.2 \(- 1 - 2 - 3 - 4\)

2.3 \(0 + 3 + 9 + 27 + 81\)

2.4 \(1 + 1 + 1 + 1 + 1\)

2.5 \(6 - 12 + 18 - 24\)

2.6 \(8 + 16 + 12 + 4\)

2.7 \(25 + 20 + 15 + 10 + 5\)

\hypertarget{q3}{%
\subsection*{Q3}\label{q3}}
\addcontentsline{toc}{subsection}{Q3}

Using the properties listed in the guide write the following sums in
their simplest form i.e.~with as little as possible within the
summation.

3.1 \[\sum_{i = 1}^{n} 2i\]

3.2 \[\sum_{i = 1}^{n} 2i + \sum_{j = 1}^{n} 2i\]

3.3 \[\sum_{i = 0}^{n} 4i + \sum_{i = 1}^{n} 2i\]

3.4 \[\sum_{i = 2}^{n} 2i - \sum_{i = 1}^{n} i\]

\hypertarget{q4}{%
\subsection*{Q4}\label{q4}}
\addcontentsline{toc}{subsection}{Q4}

Simplify the following double sums by writing them as products as sums
and applying the properties of sums

4.1 \[\sum_{i=1}^{8}\sum_{j=3}^6 3ij^2\]

4.2 \[\sum_{i=1}^{n}\sum_{j=1}^k i(j+5)\]

\hypertarget{q5}{%
\subsection*{Q5}\label{q5}}
\addcontentsline{toc}{subsection}{Q5}

Express the following as double sums in their simplest possible form

5.1 \[(3+9+27+81)(1+8+64)\]

5.2 \[\sum_{i=1}^{n}(i+2)\sum_{i=1}^{n} i\]

5.3 \[\sum_{i=1}^{n}(i+2)\sum_{i=1}^{n} i^2\]

5.4 \[\sum_{i=1}^{n}(i+2)\sum_{i=1}^{n} (i+9)\]

\hypertarget{q6}{%
\subsection*{Q6}\label{q6}}
\addcontentsline{toc}{subsection}{Q6}

Simplify the following:

6.1 \[\sum_{i = 1}^{n} (i-1)(2i+1)\]

6.2 \[\sum_{i = 1}^{n} i(i+3)(i+6)\]

6.3 \[\sum_{i = 1}^{n} i(2i-3)(3i+1)\]



\end{document}
