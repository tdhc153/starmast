% Options for packages loaded elsewhere
\PassOptionsToPackage{unicode}{hyperref}
\PassOptionsToPackage{hyphens}{url}
\PassOptionsToPackage{dvipsnames,svgnames,x11names}{xcolor}
%
\documentclass[
  12pt,
  a4paper, oneside]{starmastarticle}

\usepackage{amsmath,amssymb}
\usepackage{lmodern}
\usepackage{iftex}
\ifPDFTeX
  \usepackage[T1]{fontenc}
  \usepackage[utf8]{inputenc}
  \usepackage{textcomp} % provide euro and other symbols
\else % if luatex or xetex
  \usepackage{unicode-math}
  \defaultfontfeatures{Scale=MatchLowercase}
  \defaultfontfeatures[\rmfamily]{Ligatures=TeX,Scale=1}
\fi
% Use upquote if available, for straight quotes in verbatim environments
\IfFileExists{upquote.sty}{\usepackage{upquote}}{}
\IfFileExists{microtype.sty}{% use microtype if available
  \usepackage[]{microtype}
  \UseMicrotypeSet[protrusion]{basicmath} % disable protrusion for tt fonts
}{}
\makeatletter
\@ifundefined{KOMAClassName}{% if non-KOMA class
  \IfFileExists{parskip.sty}{%
    \usepackage{parskip}
  }{% else
    \setlength{\parindent}{0pt}
    \setlength{\parskip}{6pt plus 2pt minus 1pt}}
}{% if KOMA class
  \KOMAoptions{parskip=half}}
\makeatother
\usepackage{xcolor}
\usepackage[top=25mm,left=25mm,right=25mm,bottom=25mm]{geometry}
\setlength{\emergencystretch}{3em} % prevent overfull lines
\setcounter{secnumdepth}{-\maxdimen} % remove section numbering
% Make \paragraph and \subparagraph free-standing
\ifx\paragraph\undefined\else
  \let\oldparagraph\paragraph
  \renewcommand{\paragraph}[1]{\oldparagraph{#1}\mbox{}}
\fi
\ifx\subparagraph\undefined\else
  \let\oldsubparagraph\subparagraph
  \renewcommand{\subparagraph}[1]{\oldsubparagraph{#1}\mbox{}}
\fi


\providecommand{\tightlist}{%
  \setlength{\itemsep}{0pt}\setlength{\parskip}{0pt}}\usepackage{longtable,booktabs,array}
\usepackage{calc} % for calculating minipage widths
% Correct order of tables after \paragraph or \subparagraph
\usepackage{etoolbox}
\makeatletter
\patchcmd\longtable{\par}{\if@noskipsec\mbox{}\fi\par}{}{}
\makeatother
% Allow footnotes in longtable head/foot
\IfFileExists{footnotehyper.sty}{\usepackage{footnotehyper}}{\usepackage{footnote}}
\makesavenoteenv{longtable}
\usepackage{graphicx}
\makeatletter
\def\maxwidth{\ifdim\Gin@nat@width>\linewidth\linewidth\else\Gin@nat@width\fi}
\def\maxheight{\ifdim\Gin@nat@height>\textheight\textheight\else\Gin@nat@height\fi}
\makeatother
% Scale images if necessary, so that they will not overflow the page
% margins by default, and it is still possible to overwrite the defaults
% using explicit options in \includegraphics[width, height, ...]{}
\setkeys{Gin}{width=\maxwidth,height=\maxheight,keepaspectratio}
% Set default figure placement to htbp
\makeatletter
\def\fps@figure{htbp}
\makeatother

\usepackage{setspace}
\renewcommand{\familydefault}{cmss}
\renewcommand{\familydefault}{\sfdefault}
\usepackage{multirow}
\usepackage{colortbl}
\usepackage{fancyhdr}
\onehalfspacing
\renewcommand{\arraystretch}{1.2}
\setlength{\parskip}{0.5em}
\setlength{\parindent}{0em}
\newcommand{\mb}[1]{\mathbb{#1}} % blackboard bold
\newcommand{\mc}[1]{\mathcal{#1}} % calligraphic
\newcommand{\mf}[1]{\mathfrak{#1}} % fraktur
\newcommand{\ms}[1]{\mathscr{#1}} % script
\newcommand{\vb}[1]{\mathbf{#1}} % vector bold
\newcommand{\from}{\leftarrow}
\newcommand{\dne}{\hfill \qed \vspace{0.3cm}} % end of proof symbol
\newcommand{\abs}[1]{\left|#1\right|} % modulus signs
\newcommand{\norm}[1]{\left|\left|#1\right|\right|} % norm signs
\renewcommand{\Re}{\mathrm{Re}}
\renewcommand{\Im}{\mathrm{Im}}
\newcommand{\im}{\mathrm{im}}
\newcommand{\ds}{\displaystyle}
\makeatletter
\makeatother
\makeatletter
\makeatother
\makeatletter
\@ifpackageloaded{caption}{}{\usepackage{caption}}
\AtBeginDocument{%
\ifdefined\contentsname
  \renewcommand*\contentsname{Table of contents}
\else
  \newcommand\contentsname{Table of contents}
\fi
\ifdefined\listfigurename
  \renewcommand*\listfigurename{List of Figures}
\else
  \newcommand\listfigurename{List of Figures}
\fi
\ifdefined\listtablename
  \renewcommand*\listtablename{List of Tables}
\else
  \newcommand\listtablename{List of Tables}
\fi
\ifdefined\figurename
  \renewcommand*\figurename{Figure}
\else
  \newcommand\figurename{Figure}
\fi
\ifdefined\tablename
  \renewcommand*\tablename{Table}
\else
  \newcommand\tablename{Table}
\fi
}
\@ifpackageloaded{float}{}{\usepackage{float}}
\floatstyle{ruled}
\@ifundefined{c@chapter}{\newfloat{codelisting}{h}{lop}}{\newfloat{codelisting}{h}{lop}[chapter]}
\floatname{codelisting}{Listing}
\newcommand*\listoflistings{\listof{codelisting}{List of Listings}}
\makeatother
\makeatletter
\@ifpackageloaded{caption}{}{\usepackage{caption}}
\@ifpackageloaded{subcaption}{}{\usepackage{subcaption}}
\makeatother
\makeatletter
\@ifpackageloaded{tcolorbox}{}{\usepackage[many]{tcolorbox}}
\makeatother
\makeatletter
\@ifundefined{shadecolor}{\definecolor{shadecolor}{rgb}{.97, .97, .97}}
\makeatother
\makeatletter
\makeatother
\ifLuaTeX
  \usepackage{selnolig}  % disable illegal ligatures
\fi
\IfFileExists{bookmark.sty}{\usepackage{bookmark}}{\usepackage{hyperref}}
\IfFileExists{xurl.sty}{\usepackage{xurl}}{} % add URL line breaks if available
\urlstyle{same} % disable monospaced font for URLs
\hypersetup{
  pdftitle={Radians},
  pdfauthor={Ifan Howells-Baines, Mark Toner},
  colorlinks=true,
  linkcolor={blue},
  filecolor={Maroon},
  citecolor={Blue},
  urlcolor={Blue},
  pdfcreator={LaTeX via pandoc}}

\title{Radians}
\author{Ifan Howells-Baines, Mark Toner}
\date{}

\begin{document}
\maketitle
\begin{abstract}
tbc
\end{abstract}
\ifdefined\Shaded\renewenvironment{Shaded}{\begin{tcolorbox}[borderline west={3pt}{0pt}{shadecolor}, boxrule=0pt, breakable, frame hidden, interior hidden, enhanced, sharp corners]}{\end{tcolorbox}}\fi

\hypertarget{what-are-radians}{%
\subsection*{What are Radians?}\label{what-are-radians}}
\addcontentsline{toc}{subsection}{What are Radians?}

\textbf{Radians}, like Degrees are a way of measuring the size of angles
that can often be identified by the presence of \textbf{\(\pi\).}

Radians are derived from the \textbf{radius of a circle} which makes
them especially useful when working with circles. You will see this
relationship more in the following section.

Radians are denoted by the symbol \textbf{rad}.

This guide will firstly describe the relationship between Radians and
Circles. Then describe the relationship between Radians and Degrees
showing how to convert between them. Lastly, it will provide a table of
some useful Degree to Radians conversions for common angle sizes.

\hypertarget{radians-and-circles}{%
\subsection{Radians and circles}\label{radians-and-circles}}

There is a close relationship between radians and the circle. Let \(C\)
be the circle with center \((0,0)\) and radius \(r\). Think of taking
the line between \((r,0)\) and \((r,r)\) and bending it, without
increasing it's length or moving \((0,r)\), so that the line now covers
an arc of the circle completely. The angle between the \(x\)-axis and
where the point \((r,r)\) is after being bent is \(1\) rad.

ADD FIGURE HERE

From this, you can get the number of radians in a full circle. The
well-known equation \(c = 2 \pi r\), where \(c\) is the circumference of
a circle, tells you that you can fit arcs of length \(r\) into \(c\)
\(2\pi\) times, and since each of these segments has an angle of \(1\)
rad at the center, it follows that there are \(2\pi\) radians in the
whole circle.

ADD FIGURE HERE

\hypertarget{arc-length-and-segment-area}{%
\subsubsection{Arc Length and Segment
Area}\label{arc-length-and-segment-area}}

If you know the radius \(r\) of a circle, then you can get the arc
length \(s\) of a segment of the circle which subtends the circumference
with angle \(\theta\) by the equation \(s = r\theta\). You can also get
the area \(A\) of this segment using the equation
\(A =\frac12 r \theta^2\). Proving these equations are exercises.

ADD FIGURE HERE

\hypertarget{converting-between-radians-and-degrees}{%
\subsection{Converting between Radians and
Degrees}\label{converting-between-radians-and-degrees}}

The process for converting from Radians to Degrees and from Degrees to
Radians are direct opposites.

\hypertarget{converting-from-degrees-to-radians}{%
\subsubsection{Converting from Degrees to
Radians}\label{converting-from-degrees-to-radians}}

Starting from an angle in Degrees.

\begin{itemize}
\tightlist
\item
  Multiply the angle in Degrees by \(\pi\)
\item
  Divide the result by 180
\end{itemize}

Your angle is now in Radians.

\textbf{Remember to simplify fractions}

\hypertarget{example-1}{%
\paragraph{Example 1}\label{example-1}}

You are given the angle \(180\textdegree\) to convert to Radians.

\begin{itemize}
\tightlist
\item
  Multipying by \(\pi\) gives \(180\pi\)
\item
  Dividing by 180 gives \(\pi\)
\end{itemize}

This means that \(180\textdegree\) is equal to \(\pi\) rad.

\hypertarget{example-2}{%
\paragraph{Example 2}\label{example-2}}

You are given the angle \(45\textdegree\) to convert to Radians.

\begin{itemize}
\tightlist
\item
  Multiplying by \(\pi\) gives \(45\pi\)
\item
  Dividing by 180 gives \(\frac{\pi}{4}\)
\end{itemize}

This means that \(45\textdegree\) is equal to \(\frac{\pi}{4}\) rad.

\hypertarget{converting-from-radians-to-degrees}{%
\subsubsection{Converting from Radians to
Degrees}\label{converting-from-radians-to-degrees}}

Starting from an angle in Radians.

\begin{itemize}
\tightlist
\item
  Multiply the angle in Radians by 180
\item
  Divide the result by \(\pi\)
\end{itemize}

Your angle is now in Degrees.

\hypertarget{example-3}{%
\paragraph{Example 3}\label{example-3}}

You are given the angle \(\pi\) rad to convert to Degrees.

\begin{itemize}
\tightlist
\item
  Multiplying by 180 gives \(180\pi\)
\item
  Dividing the result by \(\pi\) gives 180
\end{itemize}

This means that \(\pi\) rad is equal to \(180\textdegree\).

\hypertarget{example-4}{%
\paragraph{Example 4}\label{example-4}}

You are given the angle \(\frac{\pi}{4}\) rad to convert to degrees

\begin{itemize}
\tightlist
\item
  Multiplying by 180 gives \(45\pi\)
\item
  Dividing the result by \(\pi\) gives 45
\end{itemize}

This means that \(\frac{\pi}{4}\) rad is equal to \(45\textdegree\).

\hypertarget{useful-conversions-to-know}{%
\subsection{Useful conversions to
know}\label{useful-conversions-to-know}}

Here is a table of useful conversions of radians to degrees that come up
often:

\begin{tabular}{|c|c|}
  \hline
  Degrees $ Radians \\
  360 & 2\pi \\
  210 & \frac{7 \pi}{6}
  180 & \pi \\
  120 & \frac{2\pi}{3}
  90 & \frac{\pi}{2}
  60 & \frac{\pi}{3}
  45 & \frac{\pi}{4}
  30 & \frac{\pi}{6}
  \hline
  
\end{tabular}



\end{document}
