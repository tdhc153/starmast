% Options for packages loaded elsewhere
\PassOptionsToPackage{unicode}{hyperref}
\PassOptionsToPackage{hyphens}{url}
\PassOptionsToPackage{dvipsnames,svgnames,x11names}{xcolor}
%
\documentclass[
  12pt,
  a4paper, oneside]{starmastarticle}

\usepackage{amsmath,amssymb}
\usepackage{lmodern}
\usepackage{iftex}
\ifPDFTeX
  \usepackage[T1]{fontenc}
  \usepackage[utf8]{inputenc}
  \usepackage{textcomp} % provide euro and other symbols
\else % if luatex or xetex
  \usepackage{unicode-math}
  \defaultfontfeatures{Scale=MatchLowercase}
  \defaultfontfeatures[\rmfamily]{Ligatures=TeX,Scale=1}
\fi
% Use upquote if available, for straight quotes in verbatim environments
\IfFileExists{upquote.sty}{\usepackage{upquote}}{}
\IfFileExists{microtype.sty}{% use microtype if available
  \usepackage[]{microtype}
  \UseMicrotypeSet[protrusion]{basicmath} % disable protrusion for tt fonts
}{}
\makeatletter
\@ifundefined{KOMAClassName}{% if non-KOMA class
  \IfFileExists{parskip.sty}{%
    \usepackage{parskip}
  }{% else
    \setlength{\parindent}{0pt}
    \setlength{\parskip}{6pt plus 2pt minus 1pt}}
}{% if KOMA class
  \KOMAoptions{parskip=half}}
\makeatother
\usepackage{xcolor}
\usepackage[top=25mm,left=25mm,right=25mm,bottom=25mm]{geometry}
\setlength{\emergencystretch}{3em} % prevent overfull lines
\setcounter{secnumdepth}{-\maxdimen} % remove section numbering
% Make \paragraph and \subparagraph free-standing
\ifx\paragraph\undefined\else
  \let\oldparagraph\paragraph
  \renewcommand{\paragraph}[1]{\oldparagraph{#1}\mbox{}}
\fi
\ifx\subparagraph\undefined\else
  \let\oldsubparagraph\subparagraph
  \renewcommand{\subparagraph}[1]{\oldsubparagraph{#1}\mbox{}}
\fi


\providecommand{\tightlist}{%
  \setlength{\itemsep}{0pt}\setlength{\parskip}{0pt}}\usepackage{longtable,booktabs,array}
\usepackage{calc} % for calculating minipage widths
% Correct order of tables after \paragraph or \subparagraph
\usepackage{etoolbox}
\makeatletter
\patchcmd\longtable{\par}{\if@noskipsec\mbox{}\fi\par}{}{}
\makeatother
% Allow footnotes in longtable head/foot
\IfFileExists{footnotehyper.sty}{\usepackage{footnotehyper}}{\usepackage{footnote}}
\makesavenoteenv{longtable}
\usepackage{graphicx}
\makeatletter
\def\maxwidth{\ifdim\Gin@nat@width>\linewidth\linewidth\else\Gin@nat@width\fi}
\def\maxheight{\ifdim\Gin@nat@height>\textheight\textheight\else\Gin@nat@height\fi}
\makeatother
% Scale images if necessary, so that they will not overflow the page
% margins by default, and it is still possible to overwrite the defaults
% using explicit options in \includegraphics[width, height, ...]{}
\setkeys{Gin}{width=\maxwidth,height=\maxheight,keepaspectratio}
% Set default figure placement to htbp
\makeatletter
\def\fps@figure{htbp}
\makeatother

\usepackage{setspace}
\renewcommand{\familydefault}{cmss}
\renewcommand{\familydefault}{\sfdefault}
\usepackage{multirow}
\usepackage{colortbl}
\usepackage{fancyhdr}
\onehalfspacing
\renewcommand{\arraystretch}{1.2}
\setlength{\parskip}{0.5em}
\setlength{\parindent}{0em}
\newcommand{\mb}[1]{\mathbb{#1}} % blackboard bold
\newcommand{\mc}[1]{\mathcal{#1}} % calligraphic
\newcommand{\mf}[1]{\mathfrak{#1}} % fraktur
\newcommand{\ms}[1]{\mathscr{#1}} % script
\newcommand{\vb}[1]{\mathbf{#1}} % vector bold
\newcommand{\from}{\leftarrow}
\newcommand{\dne}{\hfill \qed \vspace{0.3cm}} % end of proof symbol
\newcommand{\abs}[1]{\left|#1\right|} % modulus signs
\newcommand{\norm}[1]{\left|\left|#1\right|\right|} % norm signs
\renewcommand{\Re}{\mathrm{Re}}
\renewcommand{\Im}{\mathrm{Im}}
\newcommand{\im}{\mathrm{im}}
\newcommand{\ds}{\displaystyle}
\makeatletter
\@ifpackageloaded{tcolorbox}{}{\usepackage[many]{tcolorbox}}
\@ifpackageloaded{fontawesome5}{}{\usepackage{fontawesome5}}
\definecolor{quarto-callout-color}{HTML}{909090}
\definecolor{quarto-callout-note-color}{HTML}{0758E5}
\definecolor{quarto-callout-important-color}{HTML}{CC1914}
\definecolor{quarto-callout-warning-color}{HTML}{EB9113}
\definecolor{quarto-callout-tip-color}{HTML}{00A047}
\definecolor{quarto-callout-caution-color}{HTML}{FC5300}
\definecolor{quarto-callout-color-frame}{HTML}{acacac}
\definecolor{quarto-callout-note-color-frame}{HTML}{4582ec}
\definecolor{quarto-callout-important-color-frame}{HTML}{d9534f}
\definecolor{quarto-callout-warning-color-frame}{HTML}{f0ad4e}
\definecolor{quarto-callout-tip-color-frame}{HTML}{02b875}
\definecolor{quarto-callout-caution-color-frame}{HTML}{fd7e14}
\makeatother
\makeatletter
\makeatother
\makeatletter
\makeatother
\makeatletter
\@ifpackageloaded{caption}{}{\usepackage{caption}}
\AtBeginDocument{%
\ifdefined\contentsname
  \renewcommand*\contentsname{Table of contents}
\else
  \newcommand\contentsname{Table of contents}
\fi
\ifdefined\listfigurename
  \renewcommand*\listfigurename{List of Figures}
\else
  \newcommand\listfigurename{List of Figures}
\fi
\ifdefined\listtablename
  \renewcommand*\listtablename{List of Tables}
\else
  \newcommand\listtablename{List of Tables}
\fi
\ifdefined\figurename
  \renewcommand*\figurename{Figure}
\else
  \newcommand\figurename{Figure}
\fi
\ifdefined\tablename
  \renewcommand*\tablename{Table}
\else
  \newcommand\tablename{Table}
\fi
}
\@ifpackageloaded{float}{}{\usepackage{float}}
\floatstyle{ruled}
\@ifundefined{c@chapter}{\newfloat{codelisting}{h}{lop}}{\newfloat{codelisting}{h}{lop}[chapter]}
\floatname{codelisting}{Listing}
\newcommand*\listoflistings{\listof{codelisting}{List of Listings}}
\makeatother
\makeatletter
\@ifpackageloaded{caption}{}{\usepackage{caption}}
\@ifpackageloaded{subcaption}{}{\usepackage{subcaption}}
\makeatother
\makeatletter
\@ifpackageloaded{tcolorbox}{}{\usepackage[many]{tcolorbox}}
\makeatother
\makeatletter
\@ifundefined{shadecolor}{\definecolor{shadecolor}{rgb}{.97, .97, .97}}
\makeatother
\makeatletter
\makeatother
\ifLuaTeX
  \usepackage{selnolig}  % disable illegal ligatures
\fi
\IfFileExists{bookmark.sty}{\usepackage{bookmark}}{\usepackage{hyperref}}
\IfFileExists{xurl.sty}{\usepackage{xurl}}{} % add URL line breaks if available
\urlstyle{same} % disable monospaced font for URLs
\hypersetup{
  pdftitle={Trigonometry},
  pdfauthor={cc357, emg9, dtr2},
  colorlinks=true,
  linkcolor={blue},
  filecolor={Maroon},
  citecolor={Blue},
  urlcolor={Blue},
  pdfcreator={LaTeX via pandoc}}

\title{Trigonometry}
\author{cc357, emg9, dtr2}
\date{}

\begin{document}
\maketitle
\begin{abstract}
Trigonometric functions are a part of maths which will appear frequently
throughout your university career. They have uses in geometry, physics,
and much much more. This guide will look at defining the functions, how
to find the values of these functions from a given angle, some values to
remember, and some properties of the functions themselves.
\end{abstract}
\ifdefined\Shaded\renewenvironment{Shaded}{\begin{tcolorbox}[frame hidden, breakable, borderline west={3pt}{0pt}{shadecolor}, boxrule=0pt, enhanced, sharp corners, interior hidden]}{\end{tcolorbox}}\fi

\emph{Before reading this guide, you may want to review (Guide:
Radians). Additionally, after this guide you may have further questions
which can be answered by (Guide: Trigonometric Identities) or (Guide:
Solving Equations using Trigonometry and Logarithms).}

\hypertarget{what-is-trigonometry}{%
\subsection{What is trigonometry?}\label{what-is-trigonometry}}

Trigonometry deals with the relationship between the angles and the
sides of a triangle. It can be used to calculate the heights of
buildings,construct planes and even in the motion of heroes in video
games. It is mainly focused on the functions sine(sin), cosine(cos),
tangent(tan), and cotangent(cot).The first introduction you may have had
to the trigonometric functions is in the context of right angled
triangles. If you pick an angle you want to focus on, you can then label
the side opposite it with `o', the side which has no contact with the
right angle `h' (for hypotenuse), and the final side as `a' (as it is
adjacent to the angle). If you have some information about either two
sides or a side and an angle, you can use the trigonometric functions to
figure out the size of the angle or the other side. The abbrieviation
`SOH, CAH, TOA' is often used to describe this process. The first letter
in each word represents the function, the second the numerator and the
third the denominator. For example, SOH says that ``you can find
\textbf{sin} by dividing the \textbf{opposite} by the
\textbf{hypotenuse}.''

\begin{tcolorbox}[enhanced jigsaw, colback=white, opacityback=0, title=\textcolor{quarto-callout-note-color}{\faInfo}\hspace{0.5em}{Trigonometric functions}, toptitle=1mm, colbacktitle=quarto-callout-note-color!10!white, bottomrule=.15mm, left=2mm, rightrule=.15mm, breakable, colframe=quarto-callout-note-color-frame, bottomtitle=1mm, opacitybacktitle=0.6, toprule=.15mm, titlerule=0mm, arc=.35mm, coltitle=black, leftrule=.75mm]
\[\sin(\theta) = \frac{opposite\ side}{hypotenuse} \qquad\qquad \cos(\theta) = \frac{adjacent\ side}{hypotenuse}\]

\[\tan(\theta) = \frac{opposite\ side}{adjacent\ side} \qquad\qquad \cot(\theta) = \frac{adjacent\ side}{opposite\ side} \]
\end{tcolorbox}

\hypertarget{the-unit-circle}{%
\subsection*{The unit circle}\label{the-unit-circle}}
\addcontentsline{toc}{subsection}{The unit circle}

The \textbf{unit circle} uses this relationship to represent the values
of the trigonometric functions. The unit circle is a circle of radius
one which is centered on the origin of a coordinate system. Because of
this, the radius represents the hypotenuse of our right angled triangle.
As the radius/hypotenuse is one, you can use the definitions above to
see that the opposite and adjacent sides represent the absolute values
of \(sin(\theta)\) and \(cos(\theta)\) respectively. In this case,
\(\theta\) is the angle as measured anticlockwise from the positive x
axis. As such, \(\theta=0\) along the positive x axis, \(\theta=90\)
along the positive y axis, \(\theta=180\) along the negative x axis, and
\(\theta=270\) along the negative y axis. This is done in degrees,
although it may be appropriate to operate in radians, wherein it would
be 0, \(\frac{\pi}{2}\), \(/pi\), and \(\frac{3\pi}{4}\) respectively.
Whilst these angles extend beyond 180 degrees or \(\pi\) radians, which
would be an issue for triangles, you will be measuring an acute angle
that forms a triangle with one `edge' of the quadrant. This may be a
little confusing to begin with, so there will be diagrams attached to
explain. If you have any issues with these conversions, feel free to
consult the radians worksheet. You can also find \(tan(\theta)\) by
using that \(tan(\theta)=\frac{sin(\theta)}{cos(\theta)}\).

It's important to note that these are the absolute values which you get
from reading off of this system, as they are lengths. You can figure out
the sign of your functions by considering what quadrant you are looking
in; this depends on your angle, and drawing out the unit circle can
often help when starting out. Whilst I have spoken in terms of opposite
and adjacent sides of a triangle, when using the unit circle it turns
out that the magnitude of \(cos(\theta)\) will always be the x
coordinate and the magnitude of \(sin(\theta)\) always the y coordinate.
This means that \(tan(\theta)= \frac{y}{x}\) As such, you can see that
an angle over 90 degrees but less than 180 degrees will be in a quadrant
where y is positive and x is negative. This means that your value for
\(sin(\theta)\) is positive and \(cos(\theta)\) is negative. Because
they are not the same sign, \(tan(\theta)\) will also be negative.

Now the issue comes from what to do with angles greater than 360 degrees
or \(2 \pi\) radians? This can be solved by using the fact that all
trigonometric functions are periodic with a period of 360 degrees or
\(2\pi\) radians. They repeat the same pattern, and so a value of
\(sin(\theta)\) would be equal to a value of \(sin(\theta+2\pi)\) or
\(sin(\theta-6\pi)\). Keep in mind, the last one works as \(6\pi\) is an
integer multiple of \(2\pi\). It wouldn't work for \(sin(theta-3\pi)\)
for this reason.

\hypertarget{table-of-common-angles}{%
\subsection{Table of common angles}\label{table-of-common-angles}}

When you solve problems you may need different values of these
functions. To ease you mathematicians have created tables with most
frequently used values.

\begin{longtable}[]{@{}
  >{\raggedright\arraybackslash}p{(\columnwidth - 14\tabcolsep) * \real{0.1300}}
  >{\raggedright\arraybackslash}p{(\columnwidth - 14\tabcolsep) * \real{0.1300}}
  >{\raggedright\arraybackslash}p{(\columnwidth - 14\tabcolsep) * \real{0.1300}}
  >{\raggedright\arraybackslash}p{(\columnwidth - 14\tabcolsep) * \real{0.1300}}
  >{\raggedright\arraybackslash}p{(\columnwidth - 14\tabcolsep) * \real{0.1300}}
  >{\raggedright\arraybackslash}p{(\columnwidth - 14\tabcolsep) * \real{0.1300}}
  >{\raggedright\arraybackslash}p{(\columnwidth - 14\tabcolsep) * \real{0.1300}}
  >{\raggedright\arraybackslash}p{(\columnwidth - 14\tabcolsep) * \real{0.1300}}@{}}
\caption{Trgionometric values.}\tabularnewline
\toprule()
\begin{minipage}[b]{\linewidth}\raggedright
Angles(\(^\circ\))
\end{minipage} & \begin{minipage}[b]{\linewidth}\raggedright
\(0^\circ\)
\end{minipage} & \begin{minipage}[b]{\linewidth}\raggedright
\(30^\circ\)
\end{minipage} & \begin{minipage}[b]{\linewidth}\raggedright
\(45^\circ\)
\end{minipage} & \begin{minipage}[b]{\linewidth}\raggedright
\(60^\circ\)
\end{minipage} & \begin{minipage}[b]{\linewidth}\raggedright
\(90^\circ\)
\end{minipage} & \begin{minipage}[b]{\linewidth}\raggedright
\(180^\circ\)
\end{minipage} & \begin{minipage}[b]{\linewidth}\raggedright
\(270^\circ\)
\end{minipage} \\
\midrule()
\endfirsthead
\toprule()
\begin{minipage}[b]{\linewidth}\raggedright
Angles(\(^\circ\))
\end{minipage} & \begin{minipage}[b]{\linewidth}\raggedright
\(0^\circ\)
\end{minipage} & \begin{minipage}[b]{\linewidth}\raggedright
\(30^\circ\)
\end{minipage} & \begin{minipage}[b]{\linewidth}\raggedright
\(45^\circ\)
\end{minipage} & \begin{minipage}[b]{\linewidth}\raggedright
\(60^\circ\)
\end{minipage} & \begin{minipage}[b]{\linewidth}\raggedright
\(90^\circ\)
\end{minipage} & \begin{minipage}[b]{\linewidth}\raggedright
\(180^\circ\)
\end{minipage} & \begin{minipage}[b]{\linewidth}\raggedright
\(270^\circ\)
\end{minipage} \\
\midrule()
\endhead
\(\sin(\theta)\) & \(0\) & \(\frac{1}{2}\) & \(\frac{1}{\sqrt{2}}\) &
\(\frac{\sqrt{3}}{2}\) & \(1\) & \(0\) & \(-1\) \\
\(\cos(\theta)\) & \(1\) & \(\frac{\sqrt{3}}{2}\) &
\(\frac{1}{\sqrt{2}}\) & \(\frac{1}{2}\) & \(0\) & \(-1\) & \(0\) \\
\(\tan(\theta)\) & \(0\) & \(\frac{1}{\sqrt{3}}\) & \(1\) & \(\sqrt{3}\)
& \(-\) & \(0\) & \(-\) \\
\(\cot(\theta)\) & \(-\) & \(\sqrt{3}\) & \(1\) & \(\frac{1}{\sqrt{3}}\)
& \(1\) & \(-\) & \(-1\) \\
\bottomrule()
\end{longtable}

\hypertarget{properties-of-trigonometric-functions}{%
\subsection{Properties of trigonometric
functions}\label{properties-of-trigonometric-functions}}

Using the unit circle you can add an angle to your \(\theta\) and find
yourself on another point of the unit circle. You already saw that when
you move \(360^\circ\), you get to the same place. When you move
\(180^\circ\) you get to the opposite side of the unit circle. This
would lead to a change of sign as you go on the same distance, on the
other side of the axis. A more interesting transition is observed when
you shift the angle with \(90^\circ\). Then the distance that was
projected on the x-axis is now projected on the y-axis and vice versa.
This would lead to change between \(\sin\theta\) and \(\cos\theta\).
Additionally, if you look at the triangle example, you can see that the
cosine of the one not right angle is the same as the sine of the other
one. These leads to the following properties.

\begin{tcolorbox}[enhanced jigsaw, colback=white, opacityback=0, title=\textcolor{quarto-callout-note-color}{\faInfo}\hspace{0.5em}{Phase shift with \(90^\circ\)}, toptitle=1mm, colbacktitle=quarto-callout-note-color!10!white, bottomrule=.15mm, left=2mm, rightrule=.15mm, breakable, colframe=quarto-callout-note-color-frame, bottomtitle=1mm, opacitybacktitle=0.6, toprule=.15mm, titlerule=0mm, arc=.35mm, coltitle=black, leftrule=.75mm]
\[\sin(\theta+90) = \cos(\theta) \qquad \cos(\theta+90) = -\sin(\theta) \qquad \tan(\theta+90) = -\cot(\theta) \qquad \cot(\theta+90) = -\tan(\theta) \]

\[\sin(90-\theta) = \cos(\theta) \qquad \cos(90-\theta) = \sin(\theta) \qquad \tan(90-\theta) = \cot(\theta) \qquad \cot(90-\theta) = \tan(\theta) \]
\end{tcolorbox}

\hypertarget{further-reading}{%
\subsection*{Further reading}\label{further-reading}}
\addcontentsline{toc}{subsection}{Further reading}



\end{document}
