% Options for packages loaded elsewhere
\PassOptionsToPackage{unicode}{hyperref}
\PassOptionsToPackage{hyphens}{url}
\PassOptionsToPackage{dvipsnames,svgnames,x11names}{xcolor}
%
\documentclass[
  12pt,
  a4paper, oneside]{starmastarticle}

\usepackage{amsmath,amssymb}
\usepackage{iftex}
\ifPDFTeX
  \usepackage[T1]{fontenc}
  \usepackage[utf8]{inputenc}
  \usepackage{textcomp} % provide euro and other symbols
\else % if luatex or xetex
  \usepackage{unicode-math}
  \defaultfontfeatures{Scale=MatchLowercase}
  \defaultfontfeatures[\rmfamily]{Ligatures=TeX,Scale=1}
\fi
\usepackage{lmodern}
\ifPDFTeX\else  
    % xetex/luatex font selection
\fi
% Use upquote if available, for straight quotes in verbatim environments
\IfFileExists{upquote.sty}{\usepackage{upquote}}{}
\IfFileExists{microtype.sty}{% use microtype if available
  \usepackage[]{microtype}
  \UseMicrotypeSet[protrusion]{basicmath} % disable protrusion for tt fonts
}{}
\makeatletter
\@ifundefined{KOMAClassName}{% if non-KOMA class
  \IfFileExists{parskip.sty}{%
    \usepackage{parskip}
  }{% else
    \setlength{\parindent}{0pt}
    \setlength{\parskip}{6pt plus 2pt minus 1pt}}
}{% if KOMA class
  \KOMAoptions{parskip=half}}
\makeatother
\usepackage{xcolor}
\usepackage[top=25mm,left=25mm,right=25mm,bottom=25mm]{geometry}
\setlength{\emergencystretch}{3em} % prevent overfull lines
\setcounter{secnumdepth}{-\maxdimen} % remove section numbering
% Make \paragraph and \subparagraph free-standing
\ifx\paragraph\undefined\else
  \let\oldparagraph\paragraph
  \renewcommand{\paragraph}[1]{\oldparagraph{#1}\mbox{}}
\fi
\ifx\subparagraph\undefined\else
  \let\oldsubparagraph\subparagraph
  \renewcommand{\subparagraph}[1]{\oldsubparagraph{#1}\mbox{}}
\fi


\providecommand{\tightlist}{%
  \setlength{\itemsep}{0pt}\setlength{\parskip}{0pt}}\usepackage{longtable,booktabs,array}
\usepackage{calc} % for calculating minipage widths
% Correct order of tables after \paragraph or \subparagraph
\usepackage{etoolbox}
\makeatletter
\patchcmd\longtable{\par}{\if@noskipsec\mbox{}\fi\par}{}{}
\makeatother
% Allow footnotes in longtable head/foot
\IfFileExists{footnotehyper.sty}{\usepackage{footnotehyper}}{\usepackage{footnote}}
\makesavenoteenv{longtable}
\usepackage{graphicx}
\makeatletter
\def\maxwidth{\ifdim\Gin@nat@width>\linewidth\linewidth\else\Gin@nat@width\fi}
\def\maxheight{\ifdim\Gin@nat@height>\textheight\textheight\else\Gin@nat@height\fi}
\makeatother
% Scale images if necessary, so that they will not overflow the page
% margins by default, and it is still possible to overwrite the defaults
% using explicit options in \includegraphics[width, height, ...]{}
\setkeys{Gin}{width=\maxwidth,height=\maxheight,keepaspectratio}
% Set default figure placement to htbp
\makeatletter
\def\fps@figure{htbp}
\makeatother

\usepackage{setspace}
\renewcommand{\familydefault}{cmss}
\renewcommand{\familydefault}{\sfdefault}
\usepackage{multirow}
\usepackage{colortbl}
\usepackage{fancyhdr}
\onehalfspacing
\renewcommand{\arraystretch}{1.2}
\setlength{\parskip}{0.5em}
\setlength{\parindent}{0em}
\newcommand{\mb}[1]{\mathbb{#1}} % blackboard bold
\newcommand{\mc}[1]{\mathcal{#1}} % calligraphic
\newcommand{\mf}[1]{\mathfrak{#1}} % fraktur
\newcommand{\ms}[1]{\mathscr{#1}} % script
\newcommand{\vb}[1]{\mathbf{#1}} % vector bold
\newcommand{\from}{\leftarrow}
\newcommand{\dne}{\hfill \qed \vspace{0.3cm}} % end of proof symbol
\newcommand{\abs}[1]{\left|#1\right|} % modulus signs
\newcommand{\norm}[1]{\left|\left|#1\right|\right|} % norm signs
\renewcommand{\Re}{\mathrm{Re}}
\renewcommand{\Im}{\mathrm{Im}}
\newcommand{\im}{\mathrm{im}}
\newcommand{\ds}{\displaystyle}
\makeatletter
\@ifpackageloaded{tcolorbox}{}{\usepackage[skins,breakable]{tcolorbox}}
\@ifpackageloaded{fontawesome5}{}{\usepackage{fontawesome5}}
\definecolor{quarto-callout-color}{HTML}{909090}
\definecolor{quarto-callout-note-color}{HTML}{0758E5}
\definecolor{quarto-callout-important-color}{HTML}{CC1914}
\definecolor{quarto-callout-warning-color}{HTML}{EB9113}
\definecolor{quarto-callout-tip-color}{HTML}{00A047}
\definecolor{quarto-callout-caution-color}{HTML}{FC5300}
\definecolor{quarto-callout-color-frame}{HTML}{acacac}
\definecolor{quarto-callout-note-color-frame}{HTML}{4582ec}
\definecolor{quarto-callout-important-color-frame}{HTML}{d9534f}
\definecolor{quarto-callout-warning-color-frame}{HTML}{f0ad4e}
\definecolor{quarto-callout-tip-color-frame}{HTML}{02b875}
\definecolor{quarto-callout-caution-color-frame}{HTML}{fd7e14}
\makeatother
\makeatletter
\makeatother
\makeatletter
\makeatother
\makeatletter
\@ifpackageloaded{caption}{}{\usepackage{caption}}
\AtBeginDocument{%
\ifdefined\contentsname
  \renewcommand*\contentsname{Table of contents}
\else
  \newcommand\contentsname{Table of contents}
\fi
\ifdefined\listfigurename
  \renewcommand*\listfigurename{List of Figures}
\else
  \newcommand\listfigurename{List of Figures}
\fi
\ifdefined\listtablename
  \renewcommand*\listtablename{List of Tables}
\else
  \newcommand\listtablename{List of Tables}
\fi
\ifdefined\figurename
  \renewcommand*\figurename{Figure}
\else
  \newcommand\figurename{Figure}
\fi
\ifdefined\tablename
  \renewcommand*\tablename{Table}
\else
  \newcommand\tablename{Table}
\fi
}
\@ifpackageloaded{float}{}{\usepackage{float}}
\floatstyle{ruled}
\@ifundefined{c@chapter}{\newfloat{codelisting}{h}{lop}}{\newfloat{codelisting}{h}{lop}[chapter]}
\floatname{codelisting}{Listing}
\newcommand*\listoflistings{\listof{codelisting}{List of Listings}}
\makeatother
\makeatletter
\@ifpackageloaded{caption}{}{\usepackage{caption}}
\@ifpackageloaded{subcaption}{}{\usepackage{subcaption}}
\makeatother
\makeatletter
\@ifpackageloaded{tcolorbox}{}{\usepackage[skins,breakable]{tcolorbox}}
\makeatother
\makeatletter
\@ifundefined{shadecolor}{\definecolor{shadecolor}{rgb}{.97, .97, .97}}
\makeatother
\makeatletter
\makeatother
\makeatletter
\makeatother
\ifLuaTeX
  \usepackage{selnolig}  % disable illegal ligatures
\fi
\IfFileExists{bookmark.sty}{\usepackage{bookmark}}{\usepackage{hyperref}}
\IfFileExists{xurl.sty}{\usepackage{xurl}}{} % add URL line breaks if available
\urlstyle{same} % disable monospaced font for URLs
\hypersetup{
  pdftitle={Introduction to rearranging equations involving trigonometry and logarithms},
  pdfauthor={Ellie Gurini, Krish Chaudhary, Mark Toner},
  colorlinks=true,
  linkcolor={blue},
  filecolor={Maroon},
  citecolor={Blue},
  urlcolor={Blue},
  pdfcreator={LaTeX via pandoc}}

\title{Introduction to rearranging equations involving trigonometry and
logarithms}
\author{Ellie Gurini, Krish Chaudhary, Mark Toner}
\date{}

\begin{document}
\maketitle
\begin{abstract}
This guide serves to introduce rearranging equations involving
trigonometry and logarithms. This can be a useful skill, especially when
considering the uses trigonometry has in describing motion.
\end{abstract}
\ifdefined\Shaded\renewenvironment{Shaded}{\begin{tcolorbox}[frame hidden, interior hidden, enhanced, sharp corners, breakable, boxrule=0pt, borderline west={3pt}{0pt}{shadecolor}]}{\end{tcolorbox}}\fi

\emph{Before reading this, you may want to read the guides on
logarithms, trigonometry, radians and trigonometric identities.}

This guide should help to get a grasp of rearranging trigonometric and
logarithmic equations. This can be a very useful skill, especially when
considering modelling things. For example, trigonometric equations are
often used to describe signal waves and motion, whereas logarithms (or
exponentials) tend to be used when describing growth and decay.

\hypertarget{trigonometric-equations}{%
\subsection*{Trigonometric equations}\label{trigonometric-equations}}
\addcontentsline{toc}{subsection}{Trigonometric equations}

The first instance of trigonometric equations you will probably come
across will be in the form \(\cos(x)=\) a constant and you want to find
the angle x.

Let's take \(\cos(x)=0.15\) for an example. This can be done by using
your calculator to find \(\cos^{-1}(0.15)\).

These are one step problems, and you will normally have a calculator to
solve them.

If not, it's likely the angle has a commonly known value, for example
\(\cos(\frac{\pi}{2})=0\). A table of these known values is given in the
guide on Trigonometry.

An important factor to note is that throughout this guide, you will be
finding a correct answer, not the correct answer. This is due to the
periodic nature of trigonometric functions (thinking of the graph can
help visualise this- if you draw a horizontal line across the graph, it
will cross more than once!),

There are infinite answers to trigonometric equations. The way you can
find the other answers is by adding or subtracting \(2\pi\) radians or
360 degrees (or an integer multiple of) from your answer.

The next set of trigonometric equations you are likely to find are in a
similar format. This time, however, instead of just x there would be an
equation involving x. To solve this, you would solve as the first
method, then set that equal to your equation.

\begin{tcolorbox}[enhanced jigsaw, leftrule=.75mm, rightrule=.15mm, breakable, left=2mm, colback=white, bottomrule=.15mm, arc=.35mm, toprule=.15mm, opacityback=0, colframe=quarto-callout-note-color-frame]
\begin{minipage}[t]{5.5mm}
\textcolor{quarto-callout-note-color}{\faInfo}
\end{minipage}%
\begin{minipage}[t]{\textwidth - 5.5mm}

\textbf{Example 1}\vspace{2mm}

Lets say you want to solve \[\cos(3x+15)=0.5\]

A good first step is to rename \(3x+15\) to \(y\). Then, you have a
familiar equation \(\cos(y)=0.5\). You can rearrange this to
\(\cos^{-1}(0.5)=y\), which gives \(y=\frac{\pi}{3}\)

From there, you can see that \(\frac{\pi}{3}=3x+15\) from the definition
of \(y\). Please note, I've added in the step of naming y as I find it
easier to comprehend, but if you want to use \(3x+15\) the whole time,
that is also a valid method.

You can factorise the left hand side to get \(\frac{\pi}{3}= 3(x+5)\).
Then you can divide both sides of the equation by 3, so that you now
have x with no prefix, giving \(\frac{\pi}{9}= x+5\).

After this, subtract 5 from both sides to isolate x:
\(\frac{\pi}{9}-5 =x\). In some cases that will be an appropriate answer
(for example in maths or this exercise), other times you may want to
give a number (like if you are solving a physics problem). In this case
\(x= -4.65\) to 3 significant figures.

This example has a linear equation inside the trigonometric function,
but you would follow the same steps for a quadratic equation in the
function. In that case, you would just be solving where your value for
\(y\) is equal to a quadratic equation, using the methods enclosed in
the guide on Quadratic equations.

\end{minipage}%
\end{tcolorbox}

\hypertarget{quadratic-trigonometric-equations}{%
\subsection*{Quadratic trigonometric
equations}\label{quadratic-trigonometric-equations}}
\addcontentsline{toc}{subsection}{Quadratic trigonometric equations}

Another type of equation will be a quadratic equation cleverly disguised
as a trigonometric one. This is almost the opposite method from the one
just above it.

For this, you will want to treat the trigonometric function as the
variable and solve the quadratic normally, before using the method above
to find the final answer.

For example, for \(4\sin^{2}(x)+6\sin(x)+7=0\), you would solve the
equation \(4y^2+6y+7=0\) and then set \(y=\sin(x)\) and solve; by this
point, y will be a constant. For more information on solving quadratic
equations see the Quadratic Equations guide.

\hypertarget{trigonometric-equations-which-use-identities}{%
\subsection*{Trigonometric equations which use
identities}\label{trigonometric-equations-which-use-identities}}
\addcontentsline{toc}{subsection}{Trigonometric equations which use
identities}

The majority of trigonometric equations will make use of trigonometric
identities (See the guide for a list of those). For example, if you see
an equation you want to solve which includes a mixture of sin, cos, tan
etc, identities should be the first place you look in order to solve it.

\begin{tcolorbox}[enhanced jigsaw, leftrule=.75mm, rightrule=.15mm, breakable, left=2mm, colback=white, bottomrule=.15mm, arc=.35mm, toprule=.15mm, opacityback=0, colframe=quarto-callout-note-color-frame]
\begin{minipage}[t]{5.5mm}
\textcolor{quarto-callout-note-color}{\faInfo}
\end{minipage}%
\begin{minipage}[t]{\textwidth - 5.5mm}

\textbf{Example 2}\vspace{2mm}

Let's start with the equation \[4\cos^{2}(x)+6\sin^{2}(x)-6=0\]
Initially, you can't solve this equation using the methods described
above. This is where trigonometric identities play an important role.
This equation can be solved using the fact that
\(\cos^{2}(x)+\sin^2(x)=1\). The way that you know to use this identity
is that the equation involves \(\cos^2(x)\) and \(\sin^2(x)\).

Firstly, lets rewrite as \(4\cos^2(x)+4\sin^2(x)+2\sin^2(x)-6=0\) and
from there, \(4(\cos^2(x)+\sin^2(x))+2\sin^2(x)-6=0\). Then, use the
identity to say that \(4(1)+2\sin^2(x)-6=0\).

You can then move the constants to the left hand side and say that
\(2\sin^2(x)= 2\), or \(\sin^2(x)=1\). Remember that
\(\sin^2(x)= {\sin(x)}^2\)

Note that this is NOT the case for \(\sin^{-1}(x)\), as that is
describing the inverse of a function, not raising to the power of -1.

Using that, you can rewrite this as \(\sin(x)=\sqrt{1}=1\). Then, find
the inverse sin \(\sin^{-1}(1)=\frac{\pi}{2}\).

\end{minipage}%
\end{tcolorbox}

\begin{tcolorbox}[enhanced jigsaw, leftrule=.75mm, rightrule=.15mm, breakable, left=2mm, colback=white, bottomrule=.15mm, arc=.35mm, toprule=.15mm, opacityback=0, colframe=quarto-callout-note-color-frame]
\begin{minipage}[t]{5.5mm}
\textcolor{quarto-callout-note-color}{\faInfo}
\end{minipage}%
\begin{minipage}[t]{\textwidth - 5.5mm}

\textbf{Example 3}\vspace{2mm}

Prove that \(4\sec^2(x)+3= 4\tan^2(x)+7\).

To prove this you use the identity \(\sec^2(x)=\tan^2(x)+1\). You can
work from either side, but I find it easier to replace \(\sec^2(x)\),
giving the equation \(4(\tan^2(x)+1)+3\) on the left hand side. Now,
expanding the brackets should give your final answer.

\end{minipage}%
\end{tcolorbox}

\begin{tcolorbox}[enhanced jigsaw, leftrule=.75mm, rightrule=.15mm, breakable, left=2mm, colback=white, bottomrule=.15mm, arc=.35mm, toprule=.15mm, opacityback=0, colframe=quarto-callout-note-color-frame]
\begin{minipage}[t]{5.5mm}
\textcolor{quarto-callout-note-color}{\faInfo}
\end{minipage}%
\begin{minipage}[t]{\textwidth - 5.5mm}

\textbf{Example 4}\vspace{2mm}

Trigonometric identities aren't mutually exclusive. For example, you can
use that \(\sec^2(x)=1+\tan^2(x)\) to prove that
\(\cos^2(x)+\sin^2(x)=1\).

Firstly, rewrite \(\sec^2(x)\) and \(\tan^2(x)\) in terms of \(\cos(x)\)
and \(\sin(x)\). This gives the equation
\(\frac{1}{\cos^2(x)}= 1 + \frac{\sin^2(x)}{\cos^2(x)}\).

Now, you can rewrite 1 as \(\frac{\cos^2(x)}{\cos^2(x)}\), and get that
\(\frac{1}{\cos^2(x)}=\frac{\cos^2(x)+\sin^2(x)}{\cos^2(x)}\).

Finally, multiplying both sides by \(\cos^2(x)\) will leave your
identity: \(1= \cos^2(x)+\sin^2(x)\)

\end{minipage}%
\end{tcolorbox}

\hypertarget{logarithmic-equations}{%
\subsection*{Logarithmic equations}\label{logarithmic-equations}}
\addcontentsline{toc}{subsection}{Logarithmic equations}

Here's a quick reminder that you may want to reread the guide on
logarithms before starting this section.

\begin{tcolorbox}[enhanced jigsaw, bottomtitle=1mm, colbacktitle=quarto-callout-tip-color!10!white, toprule=.15mm, title=\textcolor{quarto-callout-tip-color}{\faLightbulb}\hspace{0.5em}{Tip}, colback=white, opacitybacktitle=0.6, arc=.35mm, titlerule=0mm, leftrule=.75mm, rightrule=.15mm, breakable, left=2mm, bottomrule=.15mm, toptitle=1mm, opacityback=0, coltitle=black, colframe=quarto-callout-tip-color-frame]

Remember that \(\log_{a}{b}=c\) and \(a^c=b\) are interchangeable.

\end{tcolorbox}

So one type of equation will likely be similar to \(5^x=25\). If you use
the format above, you can label a=5, b=25, and c=x. Then, you would
rewrite in the form \(\log_a{b}=\log_5{25}=x\), giving \(x=2\). This is
commonly phrased as asking a question about modelling some form of
exponential growth.

Another type may be to have an equation inside the logarithm. Lets take
a look at an example to explain this one.

\begin{tcolorbox}[enhanced jigsaw, leftrule=.75mm, rightrule=.15mm, breakable, left=2mm, colback=white, bottomrule=.15mm, arc=.35mm, toprule=.15mm, opacityback=0, colframe=quarto-callout-note-color-frame]
\begin{minipage}[t]{5.5mm}
\textcolor{quarto-callout-note-color}{\faInfo}
\end{minipage}%
\begin{minipage}[t]{\textwidth - 5.5mm}

\textbf{Example 5}\vspace{2mm}

\[\log_{10}(5x+7)=1\] You want to find x. Using the tip from above, you
can rewrite this as: \(10^1=5x+7\). From there, you can subtract 7 from
both sides, giving \(3=5x\) and then divide both sides by 5, giving
\(x=\frac{3}{5}\).

\end{minipage}%
\end{tcolorbox}

You may also come across equations which are similar to the one above,
but cleverly hidden through logarithm rules. Lets go through one of
those.

\begin{tcolorbox}[enhanced jigsaw, leftrule=.75mm, rightrule=.15mm, breakable, left=2mm, colback=white, bottomrule=.15mm, arc=.35mm, toprule=.15mm, opacityback=0, colframe=quarto-callout-note-color-frame]
\begin{minipage}[t]{5.5mm}
\textcolor{quarto-callout-note-color}{\faInfo}
\end{minipage}%
\begin{minipage}[t]{\textwidth - 5.5mm}

\textbf{Example 6}\vspace{2mm}

\[\log_{10}(4x)-\log_{10}(3)=2\]

Find x.

You start by rewriting the left hand side of the equation into one
logarithm. The guide on logarithms should have a section on this. This
will give you \(\log_{10}(\frac{4x}{3})=2\).

Now, you should rewrite this as an exponential to get
\(10^2= \frac{4x}{3}\). To finish solving, you will divide both sides by
\(\frac{4}{3}\). This gives that \(x=75\).

\end{minipage}%
\end{tcolorbox}

\begin{tcolorbox}[enhanced jigsaw, leftrule=.75mm, rightrule=.15mm, breakable, left=2mm, colback=white, bottomrule=.15mm, arc=.35mm, toprule=.15mm, opacityback=0, colframe=quarto-callout-note-color-frame]
\begin{minipage}[t]{5.5mm}
\textcolor{quarto-callout-note-color}{\faInfo}
\end{minipage}%
\begin{minipage}[t]{\textwidth - 5.5mm}

\textbf{Example 7}\vspace{2mm}

Here's a little example/proof which should help to understand the next
example more completely.

Lets have a look at an equation where a number is raised to the power of
a logarithm in its own base.

\[e^{\log_e(x)}=y\] Find y in terms of x.

Firstly, label up the equation according to the note at the start of
this. This gives a=e, b=y, c=\(\log_e(x)\). From there, rewrite the
equation as a logarithm, \(\log_e(y)=\log_e(x)\). This means that
\(y=x\). We can say this for a log function but not for, lets say, a
trigonometric function, as log functions don't have repeating values.

On a wider scale, this proves that \[a^{\log_a(b)}=b\]. Whilst e was the
base used in this example, it doesn't affect the outcome. You can repeat
this exercise with a base of 2 if you would like to confirm.

\end{minipage}%
\end{tcolorbox}

Another use of logarithms is in solving simultaneous equations. This
takes advantage of the relationship between logarithms and exponentials.

\begin{tcolorbox}[enhanced jigsaw, leftrule=.75mm, rightrule=.15mm, breakable, left=2mm, colback=white, bottomrule=.15mm, arc=.35mm, toprule=.15mm, opacityback=0, colframe=quarto-callout-note-color-frame]
\begin{minipage}[t]{5.5mm}
\textcolor{quarto-callout-note-color}{\faInfo}
\end{minipage}%
\begin{minipage}[t]{\textwidth - 5.5mm}

\textbf{Example 8}\vspace{2mm}

Solve this set of simultaneous equations \[e^y=2x+1\] \[\ln(3x)=y\] The
first step for this problem would to be to substitute \(y=\ln(3x)\) into
the first equation. This gives \(e^{\ln(3x)}=2x+1\). Now, lets label:
a=e, b=2x+1 and c=\(\ln(3x)\).

Thus, refer to our example 7 to see that \(3x=2x+1\). Subtracting 2x
from either side gives that \(x=1\). Substituting this back into our
equation gives that \(y=\ln(3)\) or approximately 1.099.

\end{minipage}%
\end{tcolorbox}

Another example of the relationship between logarithms and exponentials.

\begin{tcolorbox}[enhanced jigsaw, leftrule=.75mm, rightrule=.15mm, breakable, left=2mm, colback=white, bottomrule=.15mm, arc=.35mm, toprule=.15mm, opacityback=0, colframe=quarto-callout-note-color-frame]
\begin{minipage}[t]{5.5mm}
\textcolor{quarto-callout-note-color}{\faInfo}
\end{minipage}%
\begin{minipage}[t]{\textwidth - 5.5mm}

\textbf{Example 9}\vspace{2mm}

Solve \[5e^{-x}+3e^x=9\]

To start with this problem, you should multiply everything by \(e^x\),
giving \(5+3e^{2x}=9e^x\).

Lets rename \(e^x=y\), which should hopefully give a more familiar
equation: \(5+3y^2=9y\). You can rearrange and solve this quadratic
using the quadratic equation. This gives y=\(\frac{9+\sqrt{21}}{16}\) or
\(\frac{9-\sqrt{21}}{16}\).

\(e^x=y\) implies that \(\ln(y)=x\), so you substitute y into this
equation to give your final answer. This leaves you with
x=\(\ln{(\frac{9+\sqrt{21}}{16})}\) or
\(\ln{(\frac{9-\sqrt{21}}{16})}\), which are (to 3sf) -0.164 and -1.29
respectively.

\end{minipage}%
\end{tcolorbox}

\hypertarget{quick-check-problems}{%
\subsection*{Quick check problems}\label{quick-check-problems}}
\addcontentsline{toc}{subsection}{Quick check problems}

\begin{enumerate}
\def\labelenumi{\arabic{enumi}.}
\tightlist
\item
  You are given 3 questions and there supposed solutions. Determine if
  the solutions are True or False:
\end{enumerate}

\begin{enumerate}
\def\labelenumi{\alph{enumi})}
\item
  For \(sin(4x)= 1\) a solution is \(x =\frac{\pi}{8}\) \_\_\_\_
\item
  For \[tan(x+15)=\sqrt{3}\]
\end{enumerate}

a solution is \(x=\frac{\pi-15}{4}\) TRUE / FALSE.

\begin{enumerate}
\def\labelenumi{\alph{enumi})}
\setcounter{enumi}{2}
\tightlist
\item
  For \[cos(2x+25)= 0\]
\end{enumerate}

a solution is \(x=\frac{\pi+25}{4}\) TRUE / FALSE.

\begin{enumerate}
\def\labelenumi{\arabic{enumi}.}
\setcounter{enumi}{1}
\tightlist
\item
  Using trigonometric identities solve the following. (The expected
  identities are given in Guide: Trigonometric Identities). Please give
  angles in degrees.
\end{enumerate}

\begin{enumerate}
\def\labelenumi{\alph{enumi})}
\tightlist
\item
  \[x = 2sin^2(\theta) + 2cos^2(\theta)\]
\end{enumerate}

Answer: \(x =\) \_

\begin{enumerate}
\def\labelenumi{\alph{enumi})}
\setcounter{enumi}{1}
\item
  \[4sin^2(\theta) + 6cos^2(\theta) - 4 =0\] Answer: \(\theta =\) \_\_
\item
  \(2cot^2(\theta) = csc^2(\theta)\) Answer: \(\theta =\) \_\_
\end{enumerate}

\begin{enumerate}
\def\labelenumi{\arabic{enumi}.}
\setcounter{enumi}{2}
\tightlist
\item
  Solve the following equations for x.
\end{enumerate}

\begin{enumerate}
\def\labelenumi{\alph{enumi})}
\tightlist
\item
  \(\log_{3}{5x+4}=2\)
\end{enumerate}

Answer: \(x =\) \_.

\begin{enumerate}
\def\labelenumi{\alph{enumi})}
\setcounter{enumi}{1}
\tightlist
\item
  \(\log_{10}{5x} - \log_{10}{9} =3\)
\end{enumerate}

Answer: \(x =\) \_\_\_\_

\href{qs-rearrangingwithtrigandlogs.qmd}{For more questions on this
topic, please got o Questions: Rearranging Equations using Trigonometry
and Logarithms.}



\end{document}
