% Options for packages loaded elsewhere
\PassOptionsToPackage{unicode}{hyperref}
\PassOptionsToPackage{hyphens}{url}
\PassOptionsToPackage{dvipsnames,svgnames,x11names}{xcolor}
%
\documentclass[
  12pt,
  a4paper, oneside]{starmastarticle}

\usepackage{amsmath,amssymb}
\usepackage{lmodern}
\usepackage{iftex}
\ifPDFTeX
  \usepackage[T1]{fontenc}
  \usepackage[utf8]{inputenc}
  \usepackage{textcomp} % provide euro and other symbols
\else % if luatex or xetex
  \usepackage{unicode-math}
  \defaultfontfeatures{Scale=MatchLowercase}
  \defaultfontfeatures[\rmfamily]{Ligatures=TeX,Scale=1}
\fi
% Use upquote if available, for straight quotes in verbatim environments
\IfFileExists{upquote.sty}{\usepackage{upquote}}{}
\IfFileExists{microtype.sty}{% use microtype if available
  \usepackage[]{microtype}
  \UseMicrotypeSet[protrusion]{basicmath} % disable protrusion for tt fonts
}{}
\makeatletter
\@ifundefined{KOMAClassName}{% if non-KOMA class
  \IfFileExists{parskip.sty}{%
    \usepackage{parskip}
  }{% else
    \setlength{\parindent}{0pt}
    \setlength{\parskip}{6pt plus 2pt minus 1pt}}
}{% if KOMA class
  \KOMAoptions{parskip=half}}
\makeatother
\usepackage{xcolor}
\usepackage[top=25mm,left=25mm,right=25mm,bottom=25mm]{geometry}
\setlength{\emergencystretch}{3em} % prevent overfull lines
\setcounter{secnumdepth}{-\maxdimen} % remove section numbering
% Make \paragraph and \subparagraph free-standing
\ifx\paragraph\undefined\else
  \let\oldparagraph\paragraph
  \renewcommand{\paragraph}[1]{\oldparagraph{#1}\mbox{}}
\fi
\ifx\subparagraph\undefined\else
  \let\oldsubparagraph\subparagraph
  \renewcommand{\subparagraph}[1]{\oldsubparagraph{#1}\mbox{}}
\fi


\providecommand{\tightlist}{%
  \setlength{\itemsep}{0pt}\setlength{\parskip}{0pt}}\usepackage{longtable,booktabs,array}
\usepackage{calc} % for calculating minipage widths
% Correct order of tables after \paragraph or \subparagraph
\usepackage{etoolbox}
\makeatletter
\patchcmd\longtable{\par}{\if@noskipsec\mbox{}\fi\par}{}{}
\makeatother
% Allow footnotes in longtable head/foot
\IfFileExists{footnotehyper.sty}{\usepackage{footnotehyper}}{\usepackage{footnote}}
\makesavenoteenv{longtable}
\usepackage{graphicx}
\makeatletter
\def\maxwidth{\ifdim\Gin@nat@width>\linewidth\linewidth\else\Gin@nat@width\fi}
\def\maxheight{\ifdim\Gin@nat@height>\textheight\textheight\else\Gin@nat@height\fi}
\makeatother
% Scale images if necessary, so that they will not overflow the page
% margins by default, and it is still possible to overwrite the defaults
% using explicit options in \includegraphics[width, height, ...]{}
\setkeys{Gin}{width=\maxwidth,height=\maxheight,keepaspectratio}
% Set default figure placement to htbp
\makeatletter
\def\fps@figure{htbp}
\makeatother

\usepackage{setspace}
\renewcommand{\familydefault}{cmss}
\renewcommand{\familydefault}{\sfdefault}
\usepackage{multirow}
\usepackage{colortbl}
\usepackage{fancyhdr}
\onehalfspacing
\renewcommand{\arraystretch}{1.2}
\setlength{\parskip}{0.5em}
\setlength{\parindent}{0em}
\newcommand{\mb}[1]{\mathbb{#1}} % blackboard bold
\newcommand{\mc}[1]{\mathcal{#1}} % calligraphic
\newcommand{\mf}[1]{\mathfrak{#1}} % fraktur
\newcommand{\ms}[1]{\mathscr{#1}} % script
\newcommand{\vb}[1]{\mathbf{#1}} % vector bold
\newcommand{\from}{\leftarrow}
\newcommand{\dne}{\hfill \qed \vspace{0.3cm}} % end of proof symbol
\newcommand{\abs}[1]{\left|#1\right|} % modulus signs
\newcommand{\norm}[1]{\left|\left|#1\right|\right|} % norm signs
\renewcommand{\Re}{\mathrm{Re}}
\renewcommand{\Im}{\mathrm{Im}}
\newcommand{\im}{\mathrm{im}}
\newcommand{\ds}{\displaystyle}
\makeatletter
\@ifpackageloaded{tcolorbox}{}{\usepackage[many]{tcolorbox}}
\@ifpackageloaded{fontawesome5}{}{\usepackage{fontawesome5}}
\definecolor{quarto-callout-color}{HTML}{909090}
\definecolor{quarto-callout-note-color}{HTML}{0758E5}
\definecolor{quarto-callout-important-color}{HTML}{CC1914}
\definecolor{quarto-callout-warning-color}{HTML}{EB9113}
\definecolor{quarto-callout-tip-color}{HTML}{00A047}
\definecolor{quarto-callout-caution-color}{HTML}{FC5300}
\definecolor{quarto-callout-color-frame}{HTML}{acacac}
\definecolor{quarto-callout-note-color-frame}{HTML}{4582ec}
\definecolor{quarto-callout-important-color-frame}{HTML}{d9534f}
\definecolor{quarto-callout-warning-color-frame}{HTML}{f0ad4e}
\definecolor{quarto-callout-tip-color-frame}{HTML}{02b875}
\definecolor{quarto-callout-caution-color-frame}{HTML}{fd7e14}
\makeatother
\makeatletter
\makeatother
\makeatletter
\makeatother
\makeatletter
\@ifpackageloaded{caption}{}{\usepackage{caption}}
\AtBeginDocument{%
\ifdefined\contentsname
  \renewcommand*\contentsname{Table of contents}
\else
  \newcommand\contentsname{Table of contents}
\fi
\ifdefined\listfigurename
  \renewcommand*\listfigurename{List of Figures}
\else
  \newcommand\listfigurename{List of Figures}
\fi
\ifdefined\listtablename
  \renewcommand*\listtablename{List of Tables}
\else
  \newcommand\listtablename{List of Tables}
\fi
\ifdefined\figurename
  \renewcommand*\figurename{Figure}
\else
  \newcommand\figurename{Figure}
\fi
\ifdefined\tablename
  \renewcommand*\tablename{Table}
\else
  \newcommand\tablename{Table}
\fi
}
\@ifpackageloaded{float}{}{\usepackage{float}}
\floatstyle{ruled}
\@ifundefined{c@chapter}{\newfloat{codelisting}{h}{lop}}{\newfloat{codelisting}{h}{lop}[chapter]}
\floatname{codelisting}{Listing}
\newcommand*\listoflistings{\listof{codelisting}{List of Listings}}
\makeatother
\makeatletter
\@ifpackageloaded{caption}{}{\usepackage{caption}}
\@ifpackageloaded{subcaption}{}{\usepackage{subcaption}}
\makeatother
\makeatletter
\@ifpackageloaded{tcolorbox}{}{\usepackage[many]{tcolorbox}}
\makeatother
\makeatletter
\@ifundefined{shadecolor}{\definecolor{shadecolor}{rgb}{.97, .97, .97}}
\makeatother
\makeatletter
\makeatother
\ifLuaTeX
  \usepackage{selnolig}  % disable illegal ligatures
\fi
\IfFileExists{bookmark.sty}{\usepackage{bookmark}}{\usepackage{hyperref}}
\IfFileExists{xurl.sty}{\usepackage{xurl}}{} % add URL line breaks if available
\urlstyle{same} % disable monospaced font for URLs
\hypersetup{
  pdftitle={Sigma notation},
  pdfauthor={Tom Coleman, Ifan Howells-Baines, Mark Toner},
  colorlinks=true,
  linkcolor={blue},
  filecolor={Maroon},
  citecolor={Blue},
  urlcolor={Blue},
  pdfcreator={LaTeX via pandoc}}

\title{Sigma notation}
\author{Tom Coleman, Ifan Howells-Baines, Mark Toner}
\date{}

\begin{document}
\maketitle
\begin{abstract}
Sigma notation is used to express many additions at once. Understanding
what this notation is, how it works, and how to manipulate them is a
valuable skill to learn for use in almost any area of mathematics.
\end{abstract}
\ifdefined\Shaded\renewenvironment{Shaded}{\begin{tcolorbox}[frame hidden, borderline west={3pt}{0pt}{shadecolor}, enhanced, interior hidden, boxrule=0pt, breakable, sharp corners]}{\end{tcolorbox}}\fi

\emph{Before reading this guide, it is recommended that you read GUIDE
and GUIDE}

\hypertarget{what-is-sigma-notation}{%
\subsection*{What is sigma notation?}\label{what-is-sigma-notation}}
\addcontentsline{toc}{subsection}{What is sigma notation?}

If you want to add many things together, then it would be nice to have a
quick way of writing this down! This is where \textbf{sigma notation}
comes in.

\begin{tcolorbox}[enhanced jigsaw, bottomtitle=1mm, left=2mm, rightrule=.15mm, opacityback=0, coltitle=black, breakable, colback=white, colframe=quarto-callout-note-color-frame, leftrule=.75mm, toptitle=1mm, titlerule=0mm, title=\textcolor{quarto-callout-note-color}{\faInfo}\hspace{0.5em}{Definition of sum and sigma notation}, arc=.35mm, bottomrule=.15mm, toprule=.15mm, opacitybacktitle=0.6, colbacktitle=quarto-callout-note-color!10!white]
A \textbf{sum} is any addition of two or more real numbers. If
\(a_k,a_{k+1}, \ldots, a_n\) are real numbers (where \(k\) and \(n\) are
some natural numbers with \(k\leq N\)), then you can use \textbf{sigma
notation} to write their sum as
\[a_k + a_{k+1} + \ldots + a_N = \sum_{i = k}^N a_i\] where the right
hand side reads `the sum from \(i = k\) to \(i = n\) of the elements
\(a_i\)'. The symbol \(i\) is known as the \textbf{index} of the sum;
the index of a sum can notionally be any letter.
\end{tcolorbox}

\hypertarget{examples}{%
\paragraph*{Examples}\label{examples}}
\addcontentsline{toc}{paragraph}{Examples}

Here's some examples of sigma notation.

\begin{tcolorbox}[enhanced jigsaw, colframe=quarto-callout-note-color-frame, left=2mm, rightrule=.15mm, opacityback=0, arc=.35mm, bottomrule=.15mm, breakable, toprule=.15mm, colback=white, leftrule=.75mm]
\begin{minipage}[t]{5.5mm}
\textcolor{quarto-callout-note-color}{\faInfo}
\end{minipage}%
\begin{minipage}[t]{\textwidth - 5.5mm}
What is the value of \(\displaystyle\sum_{i=1}^{10} i\)?

You can use the definition above to write this out as a sum and then
calculate it:
\[\displaystyle\sum_{i=1}^{10} i = 1 + 2 + 3 + 4 + 5 + 6 + 7 + 8 +9 + 10 = 55.\]\end{minipage}%
\end{tcolorbox}

\begin{tcolorbox}[enhanced jigsaw, colframe=quarto-callout-note-color-frame, left=2mm, rightrule=.15mm, opacityback=0, arc=.35mm, bottomrule=.15mm, breakable, toprule=.15mm, colback=white, leftrule=.75mm]
\begin{minipage}[t]{5.5mm}
\textcolor{quarto-callout-note-color}{\faInfo}
\end{minipage}%
\begin{minipage}[t]{\textwidth - 5.5mm}
What is the value of \(\displaystyle\sum_{n=2}^5 n^2\)?

Before tackling a problem using sigma notation, it can be best to read
it out loud. Here,
\[\sum_{n=2}^5 n^2\textsf{ is 'the sum from $n = 2$ to $n = 5$ of $n^2$'.}\]
This translates to \[\sum_{n=2}^5 n^2 = 2^2 + 3^2 + 4^2 + 5^2\] and
\(2^2 + 3^2 + 4^2 + 5^2 = 4 + 9 + 16 + 25 = 54\).\end{minipage}%
\end{tcolorbox}

\begin{tcolorbox}[enhanced jigsaw, colframe=quarto-callout-note-color-frame, left=2mm, rightrule=.15mm, opacityback=0, arc=.35mm, bottomrule=.15mm, breakable, toprule=.15mm, colback=white, leftrule=.75mm]
\begin{minipage}[t]{5.5mm}
\textcolor{quarto-callout-note-color}{\faInfo}
\end{minipage}%
\begin{minipage}[t]{\textwidth - 5.5mm}
What is the value of \(\displaystyle\sum_{n=1}^N n = S\)?

In this case, you're being asked to find \(S = 1 + 2 + 3 + \ldots + N\).
The following method is due to
\href{https://mathshistory.st-andrews.ac.uk/Biographies/Gauss/}{Gauss},
who came up with this answer during a maths lesson at school when he was
seven (hinting at the genius to follow).

First of all, you can reorder \(S\) to write that
\(S = N + (N-1) + \ldots + 2 + 1\). Adding two lots of \(S\) together
gives the following:

\[
\begin{array}{cccccccccccc}
& S & = & 1 & + & 2 & + & 3 & + & \ldots & + & N \\
+ & S & = & N & + & (N-1) & + & (N-2) & + & \ldots & + & 1\\\hline
& 2S & = & (N+1) & + & (N+1) & + & (N+1) & + & \ldots & + & (N+1)
\end{array}
\]

Therefore, \(2S\) is \(N\) lots of \((N+1)\); you can write this as
\(2S = N(N+1)\). Dividing both sides by \(2\) gives
\(S = N(N+1)/2\).\end{minipage}%
\end{tcolorbox}

\hypertarget{writing-sums-using-sigma-notation}{%
\subsection{Writing sums using sigma
notation}\label{writing-sums-using-sigma-notation}}

In this section, you will learn how the opposite of the above. That is,
given a sequence of numbers, you will learn how to write their sum using
sigma notation. The crux of this process is to recognise a pattern in
the sequence of given numbers. It's best to teach this using examples:

\begin{tcolorbox}[enhanced jigsaw, colframe=quarto-callout-note-color-frame, left=2mm, rightrule=.15mm, opacityback=0, arc=.35mm, bottomrule=.15mm, breakable, toprule=.15mm, colback=white, leftrule=.75mm]
\begin{minipage}[t]{5.5mm}
\textcolor{quarto-callout-note-color}{\faInfo}
\end{minipage}%
\begin{minipage}[t]{\textwidth - 5.5mm}
Write \(2 + 4 + 6 + 8 + 10 + 12\) using sigma notation.

You can tell that these are the first six multiples of \(2\); so you can
list these elements as \(2n\) for \(n = 1\) up to \(n = 6\). Therefore,
you can write that
\[2 + 4 + 6 + 8 + 10 + 12 = \sum_{n=1}^6 2n.\]\end{minipage}%
\end{tcolorbox}

\begin{tcolorbox}[enhanced jigsaw, colframe=quarto-callout-note-color-frame, left=2mm, rightrule=.15mm, opacityback=0, arc=.35mm, bottomrule=.15mm, breakable, toprule=.15mm, colback=white, leftrule=.75mm]
\begin{minipage}[t]{5.5mm}
\textcolor{quarto-callout-note-color}{\faInfo}
\end{minipage}%
\begin{minipage}[t]{\textwidth - 5.5mm}
Write \(1 + 2 + 4 + 8 + 16 + 32\) using sigma notation.

These are the first 6 numbers in the sequence \(2^n\) for \(n=0\) up to
\(n=5\). Therefore, you can write that
\[1 + 2 + 4 + 8 + 32 = \sum_{n=0}^5 2^n.\]\end{minipage}%
\end{tcolorbox}

\begin{tcolorbox}[enhanced jigsaw, colframe=quarto-callout-note-color-frame, left=2mm, rightrule=.15mm, opacityback=0, arc=.35mm, bottomrule=.15mm, breakable, toprule=.15mm, colback=white, leftrule=.75mm]
\begin{minipage}[t]{5.5mm}
\textcolor{quarto-callout-note-color}{\faInfo}
\end{minipage}%
\begin{minipage}[t]{\textwidth - 5.5mm}
Write \(-1 + 2 -3 + 4 - 5\) using sigma notation.

For these types of sequences, it's useful to keep in mind the sequence
\((-1)^n\), which alternates between \(1\) and \(-1\). Hence, you can
write these elements as \((-1)^nn\) for \(n=1\) up to \(n=5\). Using
sigma notation, it will look like this:
\[\sum_{n=1}^5 (-1)^nn.\]\end{minipage}%
\end{tcolorbox}

\hypertarget{properties}{%
\subsection{Properties}\label{properties}}

In this section you will learn about a few nice properties of sigma
notation which means you'll have a toolkit to rearrange sums!

\begin{tcolorbox}[enhanced jigsaw, bottomtitle=1mm, left=2mm, rightrule=.15mm, opacityback=0, coltitle=black, breakable, colback=white, colframe=quarto-callout-note-color-frame, leftrule=.75mm, toptitle=1mm, titlerule=0mm, title=\textcolor{quarto-callout-note-color}{\faInfo}\hspace{0.5em}{Distribuitivity}, arc=.35mm, bottomrule=.15mm, toprule=.15mm, opacitybacktitle=0.6, colbacktitle=quarto-callout-note-color!10!white]

\end{tcolorbox}

\begin{tcolorbox}[enhanced jigsaw, bottomtitle=1mm, left=2mm, rightrule=.15mm, opacityback=0, coltitle=black, breakable, colback=white, colframe=quarto-callout-note-color-frame, leftrule=.75mm, toptitle=1mm, titlerule=0mm, title=\textcolor{quarto-callout-note-color}{\faInfo}\hspace{0.5em}{Commutativity and associativity}, arc=.35mm, bottomrule=.15mm, toprule=.15mm, opacitybacktitle=0.6, colbacktitle=quarto-callout-note-color!10!white]

\end{tcolorbox}

\hypertarget{double-sums}{%
\subsection{Double sums}\label{double-sums}}

\begin{tcolorbox}[enhanced jigsaw, colframe=quarto-callout-note-color-frame, left=2mm, rightrule=.15mm, opacityback=0, arc=.35mm, bottomrule=.15mm, breakable, toprule=.15mm, colback=white, leftrule=.75mm]
\begin{minipage}[t]{5.5mm}
\textcolor{quarto-callout-note-color}{\faInfo}
\end{minipage}%
\begin{minipage}[t]{\textwidth - 5.5mm}
A \textbf{sum} is any addition of two or more real numbers. If
\(a_k,a_{k+1}, \ldots, a_n\) are real numbers (where \(k\) and \(n\) are
some natural numbers with \(k\leq N\)), then you can use \textbf{sigma
notation} to write their sum as
\[a_k + a_{k+1} + \ldots + a_N = \sum_{i = k}^N a_i\] where the right
hand side reads `the sum from \(i = k\) to \(i = n\) of the elements
\(a_i\)'. The symbol \(i\) is known as the \textbf{index} of the sum;
the index of a sum can notionally be any letter.\end{minipage}%
\end{tcolorbox}

\hypertarget{problems}{%
\subsection{Problems}\label{problems}}

\hypertarget{further-reading}{%
\subsection{Further reading}\label{further-reading}}



\end{document}
