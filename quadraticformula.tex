% Options for packages loaded elsewhere
\PassOptionsToPackage{unicode}{hyperref}
\PassOptionsToPackage{hyphens}{url}
\PassOptionsToPackage{dvipsnames,svgnames,x11names}{xcolor}
%
\documentclass[
  12pt,
  a4paper, oneside]{starmastarticle}

\usepackage{amsmath,amssymb}
\usepackage{lmodern}
\usepackage{iftex}
\ifPDFTeX
  \usepackage[T1]{fontenc}
  \usepackage[utf8]{inputenc}
  \usepackage{textcomp} % provide euro and other symbols
\else % if luatex or xetex
  \usepackage{unicode-math}
  \defaultfontfeatures{Scale=MatchLowercase}
  \defaultfontfeatures[\rmfamily]{Ligatures=TeX,Scale=1}
\fi
% Use upquote if available, for straight quotes in verbatim environments
\IfFileExists{upquote.sty}{\usepackage{upquote}}{}
\IfFileExists{microtype.sty}{% use microtype if available
  \usepackage[]{microtype}
  \UseMicrotypeSet[protrusion]{basicmath} % disable protrusion for tt fonts
}{}
\makeatletter
\@ifundefined{KOMAClassName}{% if non-KOMA class
  \IfFileExists{parskip.sty}{%
    \usepackage{parskip}
  }{% else
    \setlength{\parindent}{0pt}
    \setlength{\parskip}{6pt plus 2pt minus 1pt}}
}{% if KOMA class
  \KOMAoptions{parskip=half}}
\makeatother
\usepackage{xcolor}
\usepackage[top=25mm,left=25mm,right=25mm,bottom=25mm]{geometry}
\setlength{\emergencystretch}{3em} % prevent overfull lines
\setcounter{secnumdepth}{-\maxdimen} % remove section numbering
% Make \paragraph and \subparagraph free-standing
\ifx\paragraph\undefined\else
  \let\oldparagraph\paragraph
  \renewcommand{\paragraph}[1]{\oldparagraph{#1}\mbox{}}
\fi
\ifx\subparagraph\undefined\else
  \let\oldsubparagraph\subparagraph
  \renewcommand{\subparagraph}[1]{\oldsubparagraph{#1}\mbox{}}
\fi


\providecommand{\tightlist}{%
  \setlength{\itemsep}{0pt}\setlength{\parskip}{0pt}}\usepackage{longtable,booktabs,array}
\usepackage{calc} % for calculating minipage widths
% Correct order of tables after \paragraph or \subparagraph
\usepackage{etoolbox}
\makeatletter
\patchcmd\longtable{\par}{\if@noskipsec\mbox{}\fi\par}{}{}
\makeatother
% Allow footnotes in longtable head/foot
\IfFileExists{footnotehyper.sty}{\usepackage{footnotehyper}}{\usepackage{footnote}}
\makesavenoteenv{longtable}
\usepackage{graphicx}
\makeatletter
\def\maxwidth{\ifdim\Gin@nat@width>\linewidth\linewidth\else\Gin@nat@width\fi}
\def\maxheight{\ifdim\Gin@nat@height>\textheight\textheight\else\Gin@nat@height\fi}
\makeatother
% Scale images if necessary, so that they will not overflow the page
% margins by default, and it is still possible to overwrite the defaults
% using explicit options in \includegraphics[width, height, ...]{}
\setkeys{Gin}{width=\maxwidth,height=\maxheight,keepaspectratio}
% Set default figure placement to htbp
\makeatletter
\def\fps@figure{htbp}
\makeatother

\usepackage{setspace}
\renewcommand{\familydefault}{cmss}
\renewcommand{\familydefault}{\sfdefault}
\usepackage{multirow}
\usepackage{colortbl}
\usepackage{fancyhdr}
\onehalfspacing
\renewcommand{\arraystretch}{1.2}
\setlength{\parskip}{0.5em}
\setlength{\parindent}{0em}
\newcommand{\mb}[1]{\mathbb{#1}} % blackboard bold
\newcommand{\mc}[1]{\mathcal{#1}} % calligraphic
\newcommand{\mf}[1]{\mathfrak{#1}} % fraktur
\newcommand{\ms}[1]{\mathscr{#1}} % script
\newcommand{\vb}[1]{\mathbf{#1}} % vector bold
\newcommand{\from}{\leftarrow}
\newcommand{\dne}{\hfill \qed \vspace{0.3cm}} % end of proof symbol
\newcommand{\abs}[1]{\left|#1\right|} % modulus signs
\newcommand{\norm}[1]{\left|\left|#1\right|\right|} % norm signs
\renewcommand{\Re}{\mathrm{Re}}
\renewcommand{\Im}{\mathrm{Im}}
\newcommand{\im}{\mathrm{im}}
\newcommand{\ds}{\displaystyle}
\makeatletter
\@ifpackageloaded{tcolorbox}{}{\usepackage[many]{tcolorbox}}
\@ifpackageloaded{fontawesome5}{}{\usepackage{fontawesome5}}
\definecolor{quarto-callout-color}{HTML}{909090}
\definecolor{quarto-callout-note-color}{HTML}{0758E5}
\definecolor{quarto-callout-important-color}{HTML}{CC1914}
\definecolor{quarto-callout-warning-color}{HTML}{EB9113}
\definecolor{quarto-callout-tip-color}{HTML}{00A047}
\definecolor{quarto-callout-caution-color}{HTML}{FC5300}
\definecolor{quarto-callout-color-frame}{HTML}{acacac}
\definecolor{quarto-callout-note-color-frame}{HTML}{4582ec}
\definecolor{quarto-callout-important-color-frame}{HTML}{d9534f}
\definecolor{quarto-callout-warning-color-frame}{HTML}{f0ad4e}
\definecolor{quarto-callout-tip-color-frame}{HTML}{02b875}
\definecolor{quarto-callout-caution-color-frame}{HTML}{fd7e14}
\makeatother
\makeatletter
\makeatother
\makeatletter
\makeatother
\makeatletter
\@ifpackageloaded{caption}{}{\usepackage{caption}}
\AtBeginDocument{%
\ifdefined\contentsname
  \renewcommand*\contentsname{Table of contents}
\else
  \newcommand\contentsname{Table of contents}
\fi
\ifdefined\listfigurename
  \renewcommand*\listfigurename{List of Figures}
\else
  \newcommand\listfigurename{List of Figures}
\fi
\ifdefined\listtablename
  \renewcommand*\listtablename{List of Tables}
\else
  \newcommand\listtablename{List of Tables}
\fi
\ifdefined\figurename
  \renewcommand*\figurename{Figure}
\else
  \newcommand\figurename{Figure}
\fi
\ifdefined\tablename
  \renewcommand*\tablename{Table}
\else
  \newcommand\tablename{Table}
\fi
}
\@ifpackageloaded{float}{}{\usepackage{float}}
\floatstyle{ruled}
\@ifundefined{c@chapter}{\newfloat{codelisting}{h}{lop}}{\newfloat{codelisting}{h}{lop}[chapter]}
\floatname{codelisting}{Listing}
\newcommand*\listoflistings{\listof{codelisting}{List of Listings}}
\makeatother
\makeatletter
\@ifpackageloaded{caption}{}{\usepackage{caption}}
\@ifpackageloaded{subcaption}{}{\usepackage{subcaption}}
\makeatother
\makeatletter
\@ifpackageloaded{tcolorbox}{}{\usepackage[many]{tcolorbox}}
\makeatother
\makeatletter
\@ifundefined{shadecolor}{\definecolor{shadecolor}{rgb}{.97, .97, .97}}
\makeatother
\makeatletter
\makeatother
\ifLuaTeX
  \usepackage{selnolig}  % disable illegal ligatures
\fi
\IfFileExists{bookmark.sty}{\usepackage{bookmark}}{\usepackage{hyperref}}
\IfFileExists{xurl.sty}{\usepackage{xurl}}{} % add URL line breaks if available
\urlstyle{same} % disable monospaced font for URLs
\hypersetup{
  pdftitle={Using the quadratic formula},
  pdfauthor={Tom Coleman},
  colorlinks=true,
  linkcolor={blue},
  filecolor={Maroon},
  citecolor={Blue},
  urlcolor={Blue},
  pdfcreator={LaTeX via pandoc}}

\title{Using the quadratic formula}
\author{Tom Coleman}
\date{}

\begin{document}
\maketitle
\begin{abstract}
Solving quadratic equations of the form \(ax^2 + bx + c\) is a core
skill in mathematics. A guaranteed method to solve these is use of the
quadratic formula. This guide explains the terminology and usage of the
quadratic formula, as well as an explanation of why the quadratic
formula works.
\end{abstract}
\ifdefined\Shaded\renewenvironment{Shaded}{\begin{tcolorbox}[sharp corners, borderline west={3pt}{0pt}{shadecolor}, frame hidden, breakable, boxrule=0pt, interior hidden, enhanced]}{\end{tcolorbox}}\fi

\emph{Before reading this guide, it is recommended that you read (Guide:
Introduction to quadratic equations) In addition, the proof of the
quadratic formula relies on the technique of `completing the square';
see (Guide: Completing the square) for more.}

\hypertarget{what-is-the-quadratic-formula}{%
\subsection*{What is the quadratic
formula?}\label{what-is-the-quadratic-formula}}
\addcontentsline{toc}{subsection}{What is the quadratic formula?}

(BLURB)

\begin{tcolorbox}[enhanced jigsaw, coltitle=black, opacitybacktitle=0.6, title=\textcolor{quarto-callout-note-color}{\faInfo}\hspace{0.5em}{The quadratic formula}, left=2mm, rightrule=.15mm, breakable, leftrule=.75mm, toprule=.15mm, opacityback=0, colbacktitle=quarto-callout-note-color!10!white, bottomtitle=1mm, arc=.35mm, toptitle=1mm, titlerule=0mm, bottomrule=.15mm, colframe=quarto-callout-note-color-frame, colback=white]
Let \(ax^2 + bx + c = 0\) be a quadratic equation (where \(a\neq 0\)).
Then the roots to this quadratic equation are given by

\[x = \frac{-b \pm \sqrt{b^2 - 4ac}}{2a}\]

where one of the roots is given by the positive term
\((-b+\sqrt{b^2 - 4ac})/2a\) and the other given by the negative term
\((-b-\sqrt{b^2 - 4ac})/2a\).
\end{tcolorbox}

\hypertarget{why-use-the-quadratic-formula}{%
\subsubsection*{Why use the quadratic
formula?}\label{why-use-the-quadratic-formula}}
\addcontentsline{toc}{subsubsection}{Why use the quadratic formula?}

(BLURB)

\hypertarget{using-the-quadratic-formula}{%
\subsection*{Using the quadratic
formula}\label{using-the-quadratic-formula}}
\addcontentsline{toc}{subsection}{Using the quadratic formula}

Here are some examples of quadratic equations which are solved using the
quadratic formula. To do this, you need to be able to identify the
variable and the coefficients \(a,b,c\).

The first of these examples illustrates the importance of making sure
the correct signs are inputted into the quadratic formula.

\begin{tcolorbox}[enhanced jigsaw, toprule=.15mm, opacityback=0, bottomrule=.15mm, arc=.35mm, left=2mm, rightrule=.15mm, breakable, leftrule=.75mm, colframe=quarto-callout-note-color-frame, colback=white]
\begin{minipage}[t]{5.5mm}
\textcolor{quarto-callout-note-color}{\faInfo}
\end{minipage}%
\begin{minipage}[t]{\textwidth - 5.5mm}
What are the roots of the quadratic equation \(x^2 - 5x + 6 = 0\)?

Here, you could factorise the quadratic equation \(x^2 - 5x + 6 = 0\) to
get \((x-2)(x-3) = 0\). This is zero precisely when \(x - 2 = 0\) or
\(x - 3 = 0\); so the roots are \(x = 2\) and \(x = 3\).

You can also work this out using the quadratic formula. In this case,
you'll need to identify the values of \(a,b,c\) in the equation
\(x^2 - 5x + 6 = 0\); here, \(a = 1\), \(b = -5\) and \(c = 6\). Taking
care of the signs, you can put these into the quadratic formula and
simplify to get

\[ x = \frac{-(-5) \pm \sqrt{(-5)^2 - 4(1)(6)}}{2(1)} = \frac{5 \pm \sqrt{25 - 24}}{2} = \frac{5 \pm \sqrt{1}}{2} = \frac{5\pm 1}{2}\]

Therefore, the roots are \(x = 4/2 = 2\) and \(x = 6/2 = 3\), as was
found above.\end{minipage}%
\end{tcolorbox}

This next example illustrates the importance of the quadratic equation
having \(0\) on one side in order to find the roots.

\begin{tcolorbox}[enhanced jigsaw, toprule=.15mm, opacityback=0, bottomrule=.15mm, arc=.35mm, left=2mm, rightrule=.15mm, breakable, leftrule=.75mm, colframe=quarto-callout-note-color-frame, colback=white]
\begin{minipage}[t]{5.5mm}
\textcolor{quarto-callout-note-color}{\faInfo}
\end{minipage}%
\begin{minipage}[t]{\textwidth - 5.5mm}
What are the roots of the quadratic equation \(-8x - 8 = x^2 + 8\)?

In order to use the quadratic formula, you need to first put the
equation into the form \(ax^2 + bx + c = 0\). So rearranging
\(-8x - 8 = x^2 + 8\) to get zero on the left hand side gives

\[0 = x^2 + 8x + 16\]

So in this case, you can take \(a = 1\), \(b = 8\) and \(c = 16\) and
put these in the quadratic formula to get

\[ x = \frac{-(8) \pm \sqrt{(8)^2 - 4(1)(16)}}{2(1)} = \frac{-8 \pm \sqrt{64 - 64}}{2} = \frac{-8 \pm \sqrt{0}}{2} = \frac{-8}{2}\]

In this case, the discriminant \(D = 0\) and so there is one distinct
root that must be counted twice. Therefore, the roots are
\(x = -8/2 = -4\) twice.\end{minipage}%
\end{tcolorbox}

This next example is the first where \(a \neq 1\); you should be
prepared to solve any quadratic equation in the form
\(ax^2 + bx + c = 0\).

\begin{tcolorbox}[enhanced jigsaw, toprule=.15mm, opacityback=0, bottomrule=.15mm, arc=.35mm, left=2mm, rightrule=.15mm, breakable, leftrule=.75mm, colframe=quarto-callout-note-color-frame, colback=white]
\begin{minipage}[t]{5.5mm}
\textcolor{quarto-callout-note-color}{\faInfo}
\end{minipage}%
\begin{minipage}[t]{\textwidth - 5.5mm}
What are the roots of the quadratic equation \(4x^2 + 4x + 5 = 0\)?

This equation is already in the form \(ax^2 + bx + c = 0\), with
\(a = 4\), \(b = 4\) and \(c = 5\). Putting these into the quadratic
formula gives

\[ x = \frac{-(4) \pm \sqrt{(4)^2 - 4(4)(5)}}{2(4)} = \frac{-4 \pm \sqrt{16 - 80}}{8} = \frac{-4 \pm \sqrt{-64}}{8}\]

In this case, the discriminant \(D < 0\). You can say that
\(\sqrt{-64} = 8i\), as \(8\) is the square root of \(64\) and \(i\) is
the square root of \(-1\). This means that

\[x = \frac{-4 \pm \sqrt{-64}}{8} = \frac{-4 \pm 8i}{8} = \frac{1}{2}\pm i\]

and these are the two roots of this quadratic equation.\end{minipage}%
\end{tcolorbox}

The next example changes the name of the variable used. You should be
able to recognise a quadratic equation in the wild, regardless of what
symbol is used as the variable.

\begin{tcolorbox}[enhanced jigsaw, toprule=.15mm, opacityback=0, bottomrule=.15mm, arc=.35mm, left=2mm, rightrule=.15mm, breakable, leftrule=.75mm, colframe=quarto-callout-note-color-frame, colback=white]
\begin{minipage}[t]{5.5mm}
\textcolor{quarto-callout-note-color}{\faInfo}
\end{minipage}%
\begin{minipage}[t]{\textwidth - 5.5mm}
What are the roots of the quadratic equation \(m^2 - m - 1 = 0\)?

Here, the quadratic equation is of the required form
\(am^2 + bm + c = 0\) with \(a = 1, b = -1\) and \(c = -1\). Putting
these into the quadratic formula gives:

\[ m = \frac{-(-1) \pm \sqrt{(-1)^2 - 4(1)(-1)}}{2(1)} = \frac{1 \pm \sqrt{1 - (-4)}}{2} = \frac{1 \pm \sqrt{5}}{2}\]

Therefore, the two roots of the quadratic equation are
\(m = (1 -\sqrt{5})/2\) and \(m = (1 + \sqrt{5})/2\). (The second of
these is a well-known mathematical constant known as the
\href{https://en.wikipedia.org/wiki/Golden_ratio}{\textbf{golden
ratio}})\end{minipage}%
\end{tcolorbox}

The final two examples deal with changes of variable and also a change
in the typical structure of a quadratic equation.

\begin{tcolorbox}[enhanced jigsaw, toprule=.15mm, opacityback=0, bottomrule=.15mm, arc=.35mm, left=2mm, rightrule=.15mm, breakable, leftrule=.75mm, colframe=quarto-callout-note-color-frame, colback=white]
\begin{minipage}[t]{5.5mm}
\textcolor{quarto-callout-note-color}{\faInfo}
\end{minipage}%
\begin{minipage}[t]{\textwidth - 5.5mm}
What are the roots of the equation \(y^4 - 3y^2 + 2 =0\)?

You can notice here that the equation given is not strictly a quadratic
equation, as it is not of the form given in the definition above.
However, by taking \(y^2\) as the variable instead of \(y\), you can
rewrite this equation as

\[y^4 - 3y^2 + 2 = (y^2)^2 - 3y^2 + 2 = 0\]

and so this \emph{is} a quadratic equation, with \(y^2\) as the
variable. You can then use the quadratic formula with \(y^2\) as the
variable, \(a = 1\), \(b = -3\) and \(c = 2\) to get

\[ y^2 = \frac{-(-3) \pm \sqrt{(-3)^2 - 4(1)(2)}}{2(1)} = \frac{3 \pm \sqrt{9 - 8}}{2} = \frac{3 \pm 1}{2}\]

Therefore, the two solutions for \(y^2\) are \(y^2 = 2/2 = 1\) and
\(y^2 = 4/2 = 2\). This gives \emph{four} solutions for \(y\), which are
\(y = 1, -1, \sqrt{2}, -\sqrt{2}\).\end{minipage}%
\end{tcolorbox}

\begin{tcolorbox}[enhanced jigsaw, toprule=.15mm, opacityback=0, bottomrule=.15mm, arc=.35mm, left=2mm, rightrule=.15mm, breakable, leftrule=.75mm, colframe=quarto-callout-note-color-frame, colback=white]
\begin{minipage}[t]{5.5mm}
\textcolor{quarto-callout-note-color}{\faInfo}
\end{minipage}%
\begin{minipage}[t]{\textwidth - 5.5mm}
What are the solutions to the equation \(4q - 13 = -3/q\)?

You can notice here that this equation does not look like a quadratic
equation at all; so some rearranging needs to be done. First of all, you
can multiply both sides by \(q\) to get

\[4q^2 - 13q = -3\]

and so

\[4q^2 - 13q + 3 = 0\]

So this really was a quadratic equation, with \(a = 4\), \(b = -13\) and
\(c = 3\). Putting this into the quadratic formula gives \begin{align*}
q &= \frac{-(-13) \pm \sqrt{(-13)^2 - 4(4)(3)}}{2(4)} = \frac{13 \pm \sqrt{169 - 48}}{8}\\[1em] 
&= \frac{13 \pm \sqrt{121}}{8} = \frac{13 \pm 11}{8}
\end{align*} Therefore, the two solutions to the equation are
\(q = 2/8 = 1/4\) and \(q = 24/8 = 3\).\end{minipage}%
\end{tcolorbox}

\hypertarget{why-does-the-quadratic-formula-work}{%
\subsection*{Why does the quadratic formula
work?}\label{why-does-the-quadratic-formula-work}}
\addcontentsline{toc}{subsection}{Why does the quadratic formula work?}

It's important to note that the quadratic formula does not come out of
thin air. In order to be able to use a result like this, you will have
to \textbf{prove} that the formula gives the roots for the quadratic
equation \(ax^2 + bx + c\).

To do this; you can \textbf{complete the square} on \(ax^2 + bx + c\)
using the fact that \(a\neq 0\). See (Guide: Completing the square) for
why this works.

\begin{tcolorbox}[enhanced jigsaw, toprule=.15mm, opacityback=0, bottomrule=.15mm, arc=.35mm, left=2mm, rightrule=.15mm, breakable, leftrule=.75mm, colframe=quarto-callout-note-color-frame, colback=white]
\begin{minipage}[t]{5.5mm}
\textcolor{quarto-callout-note-color}{\faInfo}
\end{minipage}%
\begin{minipage}[t]{\textwidth - 5.5mm}
The proof of this relies on completing the square. First of all, as
\(a\neq 0\) you can divide \(ax^2 + bx + c = 0\) through by \(a\) to get

\[x^2 + \frac{b}{a} x + \frac{c}{a} = 0\]

Taking the \(c/a\) term over to the other side gives

\[x^2 + \frac{b}{a} x = -\frac{c}{a}\]

Completing the square gives

\[\left(x + \frac{b}{2a}\right)^2 - \frac{b^2}{4a^2} = -\frac{c}{a}\]

You can rearrange to get
\[\left(x + \frac{b}{2a}\right)^2 = \frac{b^2}{4a^2} - \frac{c}{a} = \frac{b^2 - 4ac}{4a^2}\]
Now the result is starting to come together. Taking square roots of both
sides (not forgetting that it could be positive or negative) gives

\[x + \frac{b}{2a} = \pm\frac{\sqrt{b^2 - 4ac}}{2a}\]

and rearranging gives

\[x = \frac{-b \pm \sqrt{b^2 - 4ac}}{2a}\]

as required.\end{minipage}%
\end{tcolorbox}

\hypertarget{quick-check-problems}{%
\subsection*{Quick check problems}\label{quick-check-problems}}
\addcontentsline{toc}{subsection}{Quick check problems}

\hypertarget{further-reading}{%
\subsection*{Further reading}\label{further-reading}}
\addcontentsline{toc}{subsection}{Further reading}



\end{document}
